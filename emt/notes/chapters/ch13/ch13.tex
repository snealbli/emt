\documentclass[../../EMT-169.tex]{subfiles}
\documentclass[../EMT-169.tex]{subfiles}

\begin{document}
\setcounter{chapter}{0}
\label{ch:chapter1}
\clearpage

% Glossary acronym entries %
	\newacronym{ada}{ADA}{Americans With Disabilities Act}
	\newacronym{aed}{AED}{automated external defibrillator}
	\newacronym{aemt}{AEMT}{advanced EMT}
	\newacronym{als}{ALS}{advanced life support}
	\newacronym{bls}{BLS}{basic life support}
	\newacronym{cqi}{CQI}{continuous quality improvement}
	\newacronym{emd}{EMD}{emergency medical dispatch}
	\newacronym{emr}{EMR}{emergency medical responder}
	\newacronym{ems}{EMS}{emergency medical services}
	\newacronym{emt}{EMT}{emergency medical technician}
	\newacronym{hipaa}{HIPAA}{Health Insurance Portability and Accountability Act}
	\newacronym{iv}{IV}{intravenous therapy}
	\newacronym{mih}{MIH}{mobile integrated healthcare}
	\newacronym{nhtsa}{NHTSA}{National Highway Traffic Safety Administration}			% NIBAS
	\newacronym{psa}{PSA}{primary service area}
	\newacronym{qa}{QA}{quality assurance}		% NIB
	\newacronym{qc}{QC}{quality control}
	
% Glossary entries
	\newglossaryentry{advanced_EMT}
	{
		name=\acrfull{aemt},
		description={an individual who has training in specific aspects of advanced life support, such as \acrfull{iv} therapy, and the administration of certain emergency medications}
	}

	\newglossaryentry{advanced_life_support}
	{
		name=\acrfull{als},
		description={advanced lifesaving procedures, some of which are now being provided by the EMT}
	}

	\newglossaryentry{americans_with_disabilities_act}
	{
		name=\acrfull{ada},
		description={comprehensive legislation that is designed to protect people with disabilities against discrimination}
	}
	
	\newglossaryentry{automated_external_defibrillator}
	{
		name=\acrfull{aed},
		description={a device that detects treatable life-threatening cardiac dysrhythmias (ventricular fibrillation and ventricular tachycardia) and delivers the appropriate electrical shock to the patient}
	}

	\newglossaryentry{certification}
	{
		name=certification,
		description={a process in which a person, an institution, or a program is evaluated and recognized as meeting certain predetermined standards to provide safe and ethical care}
	}
	
	\newglossaryentry{community_paramedicine}
	{
		name=community paramedicine,
		description={a health care model in which experienced paramedics receive advanced training to equip them to provide additional services in the prehospital environment, such as health evaluations, monitoring of chronic illnesses or conditions, and patient advocacy}
	}

	\newglossaryentry{continuous_quality_improvement}
	{
		name=\acrfull{cqi},
		description={system of internal and external reviews and audits of all aspects of an EMS system}
	}

	\newglossaryentry{emergency_medical_dispatch}
	{
		name=\acrfull{emd},
		description={a system that assists dispatchers in selecting appropriate units to respond to a particular call for assistance and provides callers with vital instructions until the arrival of \acrshort{ems} crews}
	}

	\newglossaryentry{emergency_medical_responder}
	{
		name=\acrfull{emr},
		description={the first trained professional, such as police officers, firefighters, lifeguards, or other rescuer, to arrive at the scene of an emergency to provide initial medical assistance}
	}

	\newglossaryentry{emergency_medical_services}
	{
		name=\acrfull{ems},
		description={a multidisciplinary system that represents the combined efforts of several professionals and agencies to provide prehospital emergency care to the sick and injured}
	}

	\newglossaryentry{emergency_medical_technician}
	{
		name=\acrfull{emt},
		description={an individual who has training in basic life support, including automated external defibrillation, use of a definitive airway adjunct, and assisting patients with certain medications}
	}

	\newglossaryentry{health_insurance_portability_and_accountability_act}
	{
		name=\acrfull{hipaa},
		description={federal legislation passed in 1996; its main effect in \acrshort{ems} is in limiting the availability of patients' healthcare information and penalizing violations of patient privacy}
	}
	
	\newglossaryentry{intravenous}
	{
		name=\acrfull{iv},
		description={the delivery of a medication directly into a vein}
	}
		
	\newglossaryentry{licensure}
	{
		name=licensure,
		description={the process whereby a competent authority, usually the state, allows people to perform a regulated act}
	}
	
	\newglossaryentry{medical_control}
	{
		name=medical control,
		description={physician instructions given directly by radio or cell phone (online/direct) or indirectly by protocol/guidelines (off-line, indirect), as authorized by the medical director of the service program}
	}
	
	\newglossaryentry{medical_director}
	{
		name=medical director,
		description={the physician who authorizes or delegates to the \acrshort{emt} the authority to provide medical care in the field}
	}

	\newglossaryentry{mobile_integrated_healthcare}
	{
		name=\acrlong{mih},
		description={a method of delivering health care which involves providing health care within the community rather than at a physician's office or hospital}
	}

	\newglossaryentry{national_ems_scope_of_practice_model}
	{
		name=National \acrshort{ems} Scope of Practice Model,
		description={a document created by the \acrfull{nhtsa} that outlines the skills performed by various \acrshort{ems} providers}
	}

	\newglossaryentry{paramedic}
	{
		name=paramedic,
		description={an individual who has extensive training in \acrlong{als}, including endotracheal intubation, emergency pharmacology cardiac monitoring, and other advanced assessment and treatment skills}
	}

	\newglossaryentry{primary_prevention}
	{
		name=primary prevention,
		description={efforts to prevent an injury or illness from ever occurring}
	}

	\newglossaryentry{primary_service_area}
	{
		name=\acrlong{psa},
		description={the designated area in which the \acrshort{ems} agency is responsible for the provision prehospital emergency care and transportation to the hospital}
	}

	\newglossaryentry{public_health}
	{
		name=public health,
		description={focused on examining the health needs of entire populations with the goal of preventing health problems}
	}

	\newglossaryentry{public_safety_access_point}
	{
		name=public safety access point,
		description={a call center, staffed by trained personnel, who are responsible for managing requests for police, fire, and ambulance services}
	}

	\newglossaryentry{quality_assurance}			% NIB
	{
		name=quality assurance,
		description={See: \gls{quality_control}}
	}

	\newglossaryentry{quality_control}
	{
		name=quality control,
		description={the responsibility of the medical director to ensure the appropriate medical care standards are met the \acrshort{emt}s on each call}
	}

	\newglossaryentry{secondary_prevention}
	{
		name=secondary prevention,
		description={efforts to limit the effects of an injury or illness cannot be completely prevented} % [that you cannot completely prevent]
	}

	
\chapter{EMS Systems}

\subsection*{Abbreviations}
\begin{description}[leftmargin=!,labelwidth=\widthof{\bfseries ABCDEF}]
	\item [\acrshort{ada}] 		\acrlong{ada}
	\item [\acrshort{aed}] 		\acrlong{aed}
	\item [\acrshort{aemt}] 	\acrlong{aemt}
	\item [\acrshort{als}] 		\acrlong{als}
	\item [\acrshort{bls}] 		\acrlong{bls}
	\item [\acrshort{emd}] 		\acrlong{emd}
	\item [\acrshort{emr}] 		\acrlong{emr}
	\item [\acrshort{ems}] 		\acrlong{ems}
	\item [\acrshort{emt}] 		\acrlong{emt}
	\item [\acrshort{hipaa}]	\acrlong{hipaa}
	\item [\acrshort{iv}] 		\acrlong{iv}
	\item [\acrshort{mih}] 		\acrlong{mih}
	\item [\acrshort{nhtsa}] 	\acrlong{nhtsa}
	\item [\acrshort{psa}] 		\acrlong{psa}
	\item [\acrshort{qa}] 		\acrlong{qa}
	\item [\acrshort{qc}] 		\acrlong{qc}
\end{description}

\subsection*{Definitions}
\begin{description}	
	\item [\gls{advanced_EMT}] 						\glsdesc{advanced_EMT}
	\item [\gls{advanced_life_support}] 			\glsdesc{advanced_life_support}
	\item [\gls{americans_with_disabilities_act}] 	\glsdesc{americans_with_disabilities_act}
	\item [\gls{automated_external_defibrillator}] 	\glsdesc{automated_external_defibrillator}
	\item [\gls{certification}] 					\glsdesc{certification}
	\item [\gls{community_paramedicine}] 			\glsdesc{community_paramedicine}
	\item [\gls{continuous_quality_improvement}] 	\glsdesc{continuous_quality_improvement}
	\item [\gls{emergency_medical_dispatch}] 		\glsdesc{emergency_medical_dispatch}
	\item [\gls{emergency_medical_responder}] 		\glsdesc{emergency_medical_responder}
	\item [\gls{emergency_medical_services}] 		\glsdesc{emergency_medical_services}
	\item [\gls{emergency_medical_technician}] 		\glsdesc{emergency_medical_technician}
	\item [\gls{health_insurance_portability_and_accountability_act}] 	\glsdesc{health_insurance_portability_and_accountability_act}
	\item [\gls{intravenous}] 						\glsdesc{intravenous}
	\item [\gls{licensure}] 						\glsdesc{licensure}
	\item [\gls{medical_control}] 					\glsdesc{medical_control}
	\item [\gls{medical_director}] 					\glsdesc{medical_director}
	\item [\gls{mobile_integrated_healthcare}] 		\glsdesc{mobile_integrated_healthcare}
	\item [\gls{national_ems_scope_of_practice_model}] 		\glsdesc{national_ems_scope_of_practice_model}
	\item [\gls{paramedic}] 						\glsdesc{paramedic}
	\item [\gls{primary_prevention}] 				\glsdesc{primary_prevention}
	\item [\gls{primary_service_area}] 				\glsdesc{primary_service_area}
	\item [\gls{public_health}] 					\glsdesc{public_health}
	\item [\gls{public_safety_access_point}] 		\glsdesc{public_safety_access_point}
	\item [\gls{quality_assurance}] 				\glsdesc{quality_assurance}
	\item [\gls{quality_control}] 					\glsdesc{quality_control}
	\item [\gls{secondary_prevention}] 				\glsdesc{secondary_prevention}
\end{description} \hfill \\

\afterpage{%
\clearpage
\subsection*{Potential Test Questions}
\begin{outline}[enumerate]
	\1 What is the difference between
	\2  certification and licensure?
	\2[]
	\2 An EMT and an AEMT?
	\1[] 

	\1 What is the difference between an AEMT and an AEMT?
	\1[] 
\end{outline}
}

%\clearpage
\end{document}
\documentclass[../../EMT-169.tex]{subfiles}
\documentclass[../EMT-169.tex]{subfiles}

\begin{document}
\setcounter{chapter}{0}
\label{ch:chapter1}
\clearpage

% Glossary acronym entries %
	\newacronym{ada}{ADA}{Americans With Disabilities Act}
	\newacronym{aed}{AED}{automated external defibrillator}
	\newacronym{aemt}{AEMT}{advanced EMT}
	\newacronym{als}{ALS}{advanced life support}
	\newacronym{bls}{BLS}{basic life support}
	\newacronym{cqi}{CQI}{continuous quality improvement}
	\newacronym{emd}{EMD}{emergency medical dispatch}
	\newacronym{emr}{EMR}{emergency medical responder}
	\newacronym{ems}{EMS}{emergency medical services}
	\newacronym{emt}{EMT}{emergency medical technician}
	\newacronym{hipaa}{HIPAA}{Health Insurance Portability and Accountability Act}
	\newacronym{iv}{IV}{intravenous therapy}
	\newacronym{mih}{MIH}{mobile integrated healthcare}
	\newacronym{nhtsa}{NHTSA}{National Highway Traffic Safety Administration}			% NIBAS
	\newacronym{psa}{PSA}{primary service area}
	\newacronym{qa}{QA}{quality assurance}		% NIB
	\newacronym{qc}{QC}{quality control}
	
% Glossary entries
	\newglossaryentry{advanced_EMT}
	{
		name=\acrfull{aemt},
		description={an individual who has training in specific aspects of advanced life support, such as \acrfull{iv} therapy, and the administration of certain emergency medications}
	}

	\newglossaryentry{advanced_life_support}
	{
		name=\acrfull{als},
		description={advanced lifesaving procedures, some of which are now being provided by the EMT}
	}

	\newglossaryentry{americans_with_disabilities_act}
	{
		name=\acrfull{ada},
		description={comprehensive legislation that is designed to protect people with disabilities against discrimination}
	}
	
	\newglossaryentry{automated_external_defibrillator}
	{
		name=\acrfull{aed},
		description={a device that detects treatable life-threatening cardiac dysrhythmias (ventricular fibrillation and ventricular tachycardia) and delivers the appropriate electrical shock to the patient}
	}

	\newglossaryentry{certification}
	{
		name=certification,
		description={a process in which a person, an institution, or a program is evaluated and recognized as meeting certain predetermined standards to provide safe and ethical care}
	}
	
	\newglossaryentry{community_paramedicine}
	{
		name=community paramedicine,
		description={a health care model in which experienced paramedics receive advanced training to equip them to provide additional services in the prehospital environment, such as health evaluations, monitoring of chronic illnesses or conditions, and patient advocacy}
	}

	\newglossaryentry{continuous_quality_improvement}
	{
		name=\acrfull{cqi},
		description={system of internal and external reviews and audits of all aspects of an EMS system}
	}

	\newglossaryentry{emergency_medical_dispatch}
	{
		name=\acrfull{emd},
		description={a system that assists dispatchers in selecting appropriate units to respond to a particular call for assistance and provides callers with vital instructions until the arrival of \acrshort{ems} crews}
	}

	\newglossaryentry{emergency_medical_responder}
	{
		name=\acrfull{emr},
		description={the first trained professional, such as police officers, firefighters, lifeguards, or other rescuer, to arrive at the scene of an emergency to provide initial medical assistance}
	}

	\newglossaryentry{emergency_medical_services}
	{
		name=\acrfull{ems},
		description={a multidisciplinary system that represents the combined efforts of several professionals and agencies to provide prehospital emergency care to the sick and injured}
	}

	\newglossaryentry{emergency_medical_technician}
	{
		name=\acrfull{emt},
		description={an individual who has training in basic life support, including automated external defibrillation, use of a definitive airway adjunct, and assisting patients with certain medications}
	}

	\newglossaryentry{health_insurance_portability_and_accountability_act}
	{
		name=\acrfull{hipaa},
		description={federal legislation passed in 1996; its main effect in \acrshort{ems} is in limiting the availability of patients' healthcare information and penalizing violations of patient privacy}
	}
	
	\newglossaryentry{intravenous}
	{
		name=\acrfull{iv},
		description={the delivery of a medication directly into a vein}
	}
		
	\newglossaryentry{licensure}
	{
		name=licensure,
		description={the process whereby a competent authority, usually the state, allows people to perform a regulated act}
	}
	
	\newglossaryentry{medical_control}
	{
		name=medical control,
		description={physician instructions given directly by radio or cell phone (online/direct) or indirectly by protocol/guidelines (off-line, indirect), as authorized by the medical director of the service program}
	}
	
	\newglossaryentry{medical_director}
	{
		name=medical director,
		description={the physician who authorizes or delegates to the \acrshort{emt} the authority to provide medical care in the field}
	}

	\newglossaryentry{mobile_integrated_healthcare}
	{
		name=\acrlong{mih},
		description={a method of delivering health care which involves providing health care within the community rather than at a physician's office or hospital}
	}

	\newglossaryentry{national_ems_scope_of_practice_model}
	{
		name=National \acrshort{ems} Scope of Practice Model,
		description={a document created by the \acrfull{nhtsa} that outlines the skills performed by various \acrshort{ems} providers}
	}

	\newglossaryentry{paramedic}
	{
		name=paramedic,
		description={an individual who has extensive training in \acrlong{als}, including endotracheal intubation, emergency pharmacology cardiac monitoring, and other advanced assessment and treatment skills}
	}

	\newglossaryentry{primary_prevention}
	{
		name=primary prevention,
		description={efforts to prevent an injury or illness from ever occurring}
	}

	\newglossaryentry{primary_service_area}
	{
		name=\acrlong{psa},
		description={the designated area in which the \acrshort{ems} agency is responsible for the provision prehospital emergency care and transportation to the hospital}
	}

	\newglossaryentry{public_health}
	{
		name=public health,
		description={focused on examining the health needs of entire populations with the goal of preventing health problems}
	}

	\newglossaryentry{public_safety_access_point}
	{
		name=public safety access point,
		description={a call center, staffed by trained personnel, who are responsible for managing requests for police, fire, and ambulance services}
	}

	\newglossaryentry{quality_assurance}			% NIB
	{
		name=quality assurance,
		description={See: \gls{quality_control}}
	}

	\newglossaryentry{quality_control}
	{
		name=quality control,
		description={the responsibility of the medical director to ensure the appropriate medical care standards are met the \acrshort{emt}s on each call}
	}

	\newglossaryentry{secondary_prevention}
	{
		name=secondary prevention,
		description={efforts to limit the effects of an injury or illness cannot be completely prevented} % [that you cannot completely prevent]
	}

	
\chapter{EMS Systems}

\subsection*{Abbreviations}
\begin{description}[leftmargin=!,labelwidth=\widthof{\bfseries ABCDEF}]
	\item [\acrshort{ada}] 		\acrlong{ada}
	\item [\acrshort{aed}] 		\acrlong{aed}
	\item [\acrshort{aemt}] 	\acrlong{aemt}
	\item [\acrshort{als}] 		\acrlong{als}
	\item [\acrshort{bls}] 		\acrlong{bls}
	\item [\acrshort{emd}] 		\acrlong{emd}
	\item [\acrshort{emr}] 		\acrlong{emr}
	\item [\acrshort{ems}] 		\acrlong{ems}
	\item [\acrshort{emt}] 		\acrlong{emt}
	\item [\acrshort{hipaa}]	\acrlong{hipaa}
	\item [\acrshort{iv}] 		\acrlong{iv}
	\item [\acrshort{mih}] 		\acrlong{mih}
	\item [\acrshort{nhtsa}] 	\acrlong{nhtsa}
	\item [\acrshort{psa}] 		\acrlong{psa}
	\item [\acrshort{qa}] 		\acrlong{qa}
	\item [\acrshort{qc}] 		\acrlong{qc}
\end{description}

\subsection*{Definitions}
\begin{description}	
	\item [\gls{advanced_EMT}] 						\glsdesc{advanced_EMT}
	\item [\gls{advanced_life_support}] 			\glsdesc{advanced_life_support}
	\item [\gls{americans_with_disabilities_act}] 	\glsdesc{americans_with_disabilities_act}
	\item [\gls{automated_external_defibrillator}] 	\glsdesc{automated_external_defibrillator}
	\item [\gls{certification}] 					\glsdesc{certification}
	\item [\gls{community_paramedicine}] 			\glsdesc{community_paramedicine}
	\item [\gls{continuous_quality_improvement}] 	\glsdesc{continuous_quality_improvement}
	\item [\gls{emergency_medical_dispatch}] 		\glsdesc{emergency_medical_dispatch}
	\item [\gls{emergency_medical_responder}] 		\glsdesc{emergency_medical_responder}
	\item [\gls{emergency_medical_services}] 		\glsdesc{emergency_medical_services}
	\item [\gls{emergency_medical_technician}] 		\glsdesc{emergency_medical_technician}
	\item [\gls{health_insurance_portability_and_accountability_act}] 	\glsdesc{health_insurance_portability_and_accountability_act}
	\item [\gls{intravenous}] 						\glsdesc{intravenous}
	\item [\gls{licensure}] 						\glsdesc{licensure}
	\item [\gls{medical_control}] 					\glsdesc{medical_control}
	\item [\gls{medical_director}] 					\glsdesc{medical_director}
	\item [\gls{mobile_integrated_healthcare}] 		\glsdesc{mobile_integrated_healthcare}
	\item [\gls{national_ems_scope_of_practice_model}] 		\glsdesc{national_ems_scope_of_practice_model}
	\item [\gls{paramedic}] 						\glsdesc{paramedic}
	\item [\gls{primary_prevention}] 				\glsdesc{primary_prevention}
	\item [\gls{primary_service_area}] 				\glsdesc{primary_service_area}
	\item [\gls{public_health}] 					\glsdesc{public_health}
	\item [\gls{public_safety_access_point}] 		\glsdesc{public_safety_access_point}
	\item [\gls{quality_assurance}] 				\glsdesc{quality_assurance}
	\item [\gls{quality_control}] 					\glsdesc{quality_control}
	\item [\gls{secondary_prevention}] 				\glsdesc{secondary_prevention}
\end{description} \hfill \\

\afterpage{%
\clearpage
\subsection*{Potential Test Questions}
\begin{outline}[enumerate]
	\1 What is the difference between
	\2  certification and licensure?
	\2[]
	\2 An EMT and an AEMT?
	\1[] 

	\1 What is the difference between an AEMT and an AEMT?
	\1[] 
\end{outline}
}

%\clearpage
\end{document}
\documentclass[../../EMT-169.tex]{subfiles}

\begin{document}
\setcounter{chapter}{1}
\label{ch:chapter2}
\clearpage
	
	
% Glossary acronym entries %
	\newacronym{aids}{AIDS}{acquired immunodeficiency syndrome}
	\newacronym{cdc}{CDC}{Center for Disease Control and Prevention}
	\newacronym{cism}{CISM}{critical incident stress management system}
	\newacronym{hiv}{HIV}{human immunodeficiency virus}
	\newacronym{osha}{OSHA}{Occupational Safety and Health Administration}
	\newacronym{ppe}{PPE}{personal protective equipment}
	\newacronym{ptsd}{PTSD}{posttraumatic stress disorder}
	
% Glossary entries
	\newglossaryentry{acute_stress_reactions}
	{
		name=acute stress reactions,
		description={reactions to stress that occur during a stressful situation}
	}
	
	\newglossaryentry{airborne_transmission}
	{
		name=airborne transmission,
		description={the spread of an organism via droplets or dust}
	}
	
	\newglossaryentry{blood_borne_pathogens}
	{
		name=blood-borne pathogens,
		description={pathogenic microorganisms that are present in human blood and can cause disease in humans.  These pathogens include, but are not limited to, hepatitis B virus and human immunodeficiency virus (HIV)}
	}
	
	\newglossaryentry{centers_for_disease_control_and_prevention}
	{
		name=\acrfull{cdc},
		description={the primary federal agency that conducts and supports public health activities in the United States.  The CDC is part of the US Department of Health and Human Services}
	}
	
	\newglossaryentry{communicable_disease}
	{
		name=communicable disease,
		description={a disease that can be spread from one person or species to another}
	}
	
	\newglossaryentry{concealment}
	{
		name=concealment,
		description={the use of objects to limit a person’s visibility of you}
	}
	
	\newglossaryentry{contamination}
	{
		name=contamination,
		description={the presence of infectious organisms on or in objects such as dressings, water, food, needles, wounds, or patient’s body}
	}
	
	\newglossaryentry{cover}
	{
		name=cover,
		description={the tactical use of an impenetrable barrier for protection}
	}
	
	\newglossaryentry{critical_incident_stress_management_system}
	{
		name=\acrfull{cism},
		description={a process that confronts the responses to critical incidents and defuses them, directing the emergency services personnel toward physical and emotional equilibrium}
	}
	
	\newglossaryentry{cumulative_stress_reactions}
	{
		name=cumulative stress reactions,
		description={prolonged or excessive stress}
	}
	
	\newglossaryentry{delayed_stress_reactions}
	{
		name=delayed stress reactions,
		description={reactions to stress that occur after a stressful situation}
	}
	
	\newglossaryentry{designated_officer}
	{
		name=designated officer,
		description={the individual in the department who is charged with the responsibility of managing exposures and infection control issues}
	}
	
	\newglossaryentry{direct_contact}
	{
		name=direct contact,
		description={exposure a transmission of a communicable disease from one person to another by physical contact}
	}
	
	\newglossaryentry{exposure}
	{
		name=exposure,
		description={a situation in which a person has contact with blood, body fluids, tissues, or airborne particles in a matter that suggest disease transmission may occur}
	}
	
	\newglossaryentry{foodborne_transmission}
	{
		name=foodborne transmission,
		description={the contamination of food or water with an organism that can cause disease}
	}
	
	\newglossaryentry{general_adaptation_syndrome}
	{
		name=general adaptation syndrome,
		description={the body’s response to stress that begins with alarm response, followed by a stage of reaction and resistance, then recovery or, if distress is prolonged, exhaustion}
	}
	
	\newglossaryentry{hepatitis}
	{
		name=hepatitis,
		description={inflammation of delivers, usually caused by viral infection, a causes fever, loss of appetite, jaundice, fatigue, and altered liver function}
	}
	
	\newglossaryentry{host}
	{
		name=host,
		description={the organism or individual is attacked by the infecting agent}
	}
	
	\newglossaryentry{human_immunodeficiency_virus}
	{
		name=\acrfull{hiv},
		description={\acrfull{aids} is caused by HIV, which damages the cells in the body’s immune system so that the body is unable to fight infection or certain cancers}
	}
	
	\newglossaryentry{immune}
	{
		name=immune,
		description={the body’s ability to protect itself from acquiring a disease}
	}
	
	\newglossaryentry{indirect_contact}
	{
		name=indirect contact,
		description={exposure or transmission of a disease from one person to another by contact with a contaminated object}
	}
	
	\newglossaryentry{infection}
	{
		name=infection,
		description={the abnormal invasion of a host or host tissues by organisms such as bacteria, viruses, or parasites, with or without signs or symptoms of disease}
	}
	
	\newglossaryentry{infection_control}
	{
		name=infection control,
		description={procedures to reduce transmission of infection among patients and healthcare personnel}
	}
	
	\newglossaryentry{infectious_disease}
	{
		name=infectious disease,
		description={a medical condition caused by the growth and spread of small, harmful organisms within the body}
	}
	
	\newglossaryentry{occupational_safety_and_health_administration}
	{
		name=\acrfull{osha},
		description={the federal regulatory compliance agency that develops, publishes, and enforces guidelines concerning safety in the workplace}
	}
	
	\newglossaryentry{pathogen}
	{
		name=pathogen,
		description={a microorganism that is capable of causing disease in a susceptible host}
	}
	
	\newglossaryentry{personal_protective_equipment}
	{
		name=\acrfull{ppe},
		description={protective equipment that blocks exposure to a pathogen or a hazardous material}
	}
	
	\newglossaryentry{posttraumatic_stress_disorder}
	{
		name=\acrfull{ptsd},
		description={a delayed stress reaction to a prior incident.  Often the result of one or more unresolved issues concerning the incident, and may relate to an incident that involved physical harm or the threat of physical harm}
	}
	
	\newglossaryentry{transmission}
	{
		name=transmission,
		description={the way in which an infectious disease is spread: contact, airborne, by vehicles, or by vectors}
	}
	
	\newglossaryentry{standard_precautions}
	{
		name=standard precautions,
		description={protective measures that have traditionally been developed by the \acrshort{cdc} for use in dealing with objects, blood, body fluids, and other potential exposure risks of communicable disease}
	}
	
	\newglossaryentry{vector_borne_transmission}
	{
		name=vector-borne transmission,
		description={the use of an animal to spread an organism from one person or place to another}
	}
	

\chapter{Workforce Safety and Wellness}

\subsection*{Abbreviations}
\begin{description}[leftmargin=!,labelwidth=\widthof{\bfseries ABCDE}]
	\item [\acrshort{aids}] 	\acrlong{aids}
	\item [\acrshort{cdc}] 		\acrlong{cdc}
	\item [\acrshort{cism}] 	\acrlong{cism}
	\item [\acrshort{hiv}] 		\acrlong{hiv}
	\item [\acrshort{osha}] 	\acrlong{osha}
	\item [\acrshort{ppe}] 		\acrlong{ppe}
	\item [\acrshort{ptsd}] 	\acrlong{ptsd}
\end{description}

\subsection*{Definitions}
\begin{description}
	\item [\gls{acute_stress_reactions}] 			\glsdesc{acute_stress_reactions}
	\item [\gls{airborne_transmission}] 			\glsdesc{airborne_transmission}
	\item [\gls{blood_borne_pathogens}] 			\glsdesc{blood_borne_pathogens}
	\item [\gls{centers_for_disease_control_and_prevention}] 	\glsdesc{centers_for_disease_control_and_prevention}
	\item [\gls{communicable_disease}] 				\glsdesc{communicable_disease}
	\item [\gls{concealment}] 						\glsdesc{concealment}
	\item [\gls{contamination}] 					\glsdesc{contamination}
	\item [\gls{cover}] 							\glsdesc{cover}
	\item [\gls{critical_incident_stress_management_system}] 	\glsdesc{critical_incident_stress_management_system}
	\item [\gls{cumulative_stress_reactions}] 		\glsdesc{cumulative_stress_reactions}
	\item [\gls{delayed_stress_reactions}] 			\glsdesc{delayed_stress_reactions}
	\item [\gls{designated_officer}] 				\glsdesc{designated_officer}
	\item [\gls{direct_contact}] 					\glsdesc{direct_contact}
	\item [\gls{exposure}] 							\glsdesc{exposure}
	\item [\gls{foodborne_transmission}] 			\glsdesc{foodborne_transmission}
	\item [\gls{general_adaptation_syndrome}] 		\glsdesc{general_adaptation_syndrome}
	\item [\gls{hepatitis}] 						\glsdesc{hepatitis}
	\item [\gls{host}] 								\glsdesc{host}
	\item [\gls{human_immunodeficiency_virus}] 		\glsdesc{human_immunodeficiency_virus}
	\item [\gls{immune}] 							\glsdesc{immune}
	\item [\gls{indirect_contact}] 					\glsdesc{indirect_contact}
	\item [\gls{infection}] 						\glsdesc{infection}
	\item [\gls{infection_control}] 				\glsdesc{infection_control}
	\item [\gls{infectious_disease}] 				\glsdesc{infectious_disease}
	\item [\gls{occupational_safety_and_health_administration}] 	\glsdesc{occupational_safety_and_health_administration}
	\item [\gls{pathogen}] 							\glsdesc{pathogen}
	\item [\gls{personal_protective_equipment}] 	\glsdesc{personal_protective_equipment}
	\item [\gls{posttraumatic_stress_disorder}] 	\glsdesc{posttraumatic_stress_disorder}
	\item [\gls{transmission}] 						\glsdesc{transmission}
	\item [\gls{standard_precautions}] 				\glsdesc{standard_precautions}
	\item [\gls{vector_borne_transmission}] 		\glsdesc{vector_borne_transmission}
\end{description}\hfill \\

%\afterpage{%
%\clearpage
%\subsection{Potential Test Questions}
%\begin{outline}[enumerate]
%	\1 Why do we need a template here?
%	\1[] Because LaTeX is stupid so we have to.
%\end{outline}
%}

\clearpage
\end{document}
\documentclass[../../EMT-169.tex]{subfiles}


\begin{document}
\setcounter{chapter}{5}
\label{ch:chapter6}
\clearpage

% Glossary acronym entries %
	\newacronym{ans}{ANS}{autonomic nervous system}
	\newacronym{atp}{ATP}{adenosine triphospate}
	\newacronym{bp}{BP}{blood pressure}
	\newacronym{co}{CO}{cardiac output}
	\newacronym{csf}{CSF}{cerebrospinal fluid}
	\newacronym{cns}{cns}{central nervous system}
	\newacronym{ens}{SNS}{enteric nervous system}
	\newacronym{hr}{HR}{heart rate}
	\newacronym{pns}{PNS}{peripheral nervous system}
	\newacronym{psns}{PSNS}{parasympathetic nervous system}
	\newacronym{sns}{SNS}{sympathetic nervous system}
	\newacronym{sv}{SV}{stroke volume}
	\newacronym{svr}{SVR}{systemic vascular resistance}

% Glossary entries
	\newglossaryentry{abdomen}
	{
		name=abdomen,
		description={the body cavity that contains the major organs of digestion and excretion. It is located below the diaphragm and above the pelvis}
	}

	\newglossaryentry{accessory_muscles}
	{
		name=accessory muscles,
		description={the secondary muscles of respiration.  They include the neck muscles (sternocleidomastoids), the chest pecoralis major muscles, and the abdominal muscles}
	}

	\newglossaryentry{acetabulum}
	{
		name=acetabulum,
		description={depression on the lateral pelvis where the three component bones join, in which the femoral head fits snugly}
	}
	
	\newglossaryentry{adams_apple}
	{
		name=Adam's apple,
		description={a firm prominence of cartilage that forms the upper part of the larynx. It is more prominent in men than in women}
	}

	\newglossaryentry{adenosine_triphosphate}
	{
		name=\acrlong{atp},
		description={the nucleotide involved in energy metabolism; used to store energy}
	}
	
	\newglossaryentry{adrenal_gland}
	{
		name=adrenal gland,
		description={endocrine gland located on top of each kidney that releases adrenaline when stimulated by the \acrlong{sns}}
	}
	
	\newglossaryentry{adrenaline}
	{
		name=adrenaline,
		description={another name for epinephrine}
	}
	
	\newglossaryentry{adrenergic}
	{
		name=adrenergic,
		description={pertaining to nerves that release the neurotransmitter norepinephrine, or noradrenalin (such as adrenergic nerves, adrenergic response); also pertains to the receptors acted on by norepinephrine}
	}
	
	\newglossaryentry{aerobic_metabolism}
	{
		name=aerobic metabolism,
		description={metabolism the can proceed only in the presence of oxygen}
	}
	
	\newglossaryentry{agonal_gasps}
	{
		name=agonal gasps,
		description={abnormal breathing pattern characterized by slow, gasping breaths, sometimes seen in patients in cardiac arrest}
	}
	
	\newglossaryentry{alpha_adrenergic_receptors}
	{
		name=alpha-adrenergic receptors,
		description={portions of the nervous system that, when stimulated, can cause constriction of the blood vessels}
	}
	
	\newglossaryentry{alveoli}
	{
		name=alveoli,
		description={air sacs of the lungs in which the exchange of oxygen and carbon dioxide takes place}
	}
	
	\newglossaryentry{anaerobic_metabolism}
	{
		name=anaerobic metabolism,
		description={the metabolism that takes place in the absence of oxygen; the main byproduct is lactic acid}
	}
	
	\newglossaryentry{anatomic_position}
	{
		name=anatomic position,
		description={the position of reference in which the patient stands facing forward, arms at the side, with the palms of the hands forward}
	}

	\newglossaryentry{aorta}
	{
		name=aorta,
		description={the main artery that receives blood from the left ventricle and delivers it to all the other arteries that carry blood to the tissues of the body}
	}

	\newglossaryentry{appendicular_skeleton}
	{
		name=appendicular skeleton,
		description={the portion of the skeletal system that comprises the arms, legs, pelvis, and shoulder girdle}
	}
	
	\newglossaryentry{appendix}
	{
		name=appendix,
		description={a small, tubular structure that is attached to the lower border of the cecum in the lower right quadrant of the abdomen}
	}

	\newglossaryentry{arteriole}
	{
		name=arteriole,
		description={the smallest branch of arteries leading to the vast network of capillaries}
	}

	\newglossaryentry{artery}
	{
		name=artery,
		description={a blood vessel, consisting of three layers of tissue and smooth muscle, that carries blood away from the heart}
	}

	\newglossaryentry{articular_cartilage}
	{
		name=articular cartilage,
		description={a pearly layer specialized cartilage covering the articular surfaces (contact surfaces on the ends) of bones in synovial joints}
	}

	\newglossaryentry{atrium}
	{
		name=atrium ,
		description={one of the two upper chambers of the heart}
	}

	\newglossaryentry{autonomic_nervous_system}
	{
		name=\acrlong{ans},
		description={division of \acrlong{pns}; regulates involuntary activities of the body such as heart rate blood pressure and digestion of food}
	}

	\newglossaryentry{axial_plane}
	{
		name=axial plane,
		description={see: \gls{transverse_plane}}
	}

	\newglossaryentry{axial_skeleton}
	{
		name=axial skeleton,
		description={the part of the skull to comprising the skull, spinal column, and rib cage}
	}

	\newglossaryentry{ball_and_socket_joint}
	{
		name=ball-and-socket joint,
		description={a joint that allows internal and external rotation, as well as bending}
	}
	
	\newglossaryentry{beta_adrenergic_receptors}
	{
		name=beta-adrenergic receptors,
		description={portions of the nervous system that, when stimulated, can cause an increase in the force of contraction of the heart, an increased heart rate, and bronchial dilation}
	}
	
	\newglossaryentry{biceps}
	{
		name=biceps,
		description={the large muscles that cover the front of the humerus}
	}
	
	\newglossaryentry{bile_duct}
	{
		name=bile duct,
		description={the duct that conveys bile between the liver and the intestine}
	}
	
	\newglossaryentry{blood_pressure}
	{
		name= \acrlong{bp},
		description={pressure that the blood exerts against the walls of the arteries as it passes through them}
	}
	
	\newglossaryentry{brachial_artery}
	{
		name=brachial artery,
		description={the major blood vessel in the upper extremities that supplies blood to the arm}
	}
	
	\newglossaryentry{brain}
	{
		name=brain,
		description={the controlling organ of the body and center of consciousness; functions include perception, control of reactions to the environment, emotional responses, and judgment}
	}
	
	\newglossaryentry{brainstem}
	{
		name=brainstem,
		description={the area of the brain between the spinal cord and cerebrum, surrounded by the cerebellum; controls functions that are necessary for life, such as respiration}
	}

	\newglossaryentry{capillary}
	{
		name=capillary,
		description={a small blood vessel that connects arterials and venules; various substances pass through capillary walls, into and out of the interstitial fluid, and then on to the cells}
	}

	\newglossaryentry{capillary_vessels}
	{
		name=capillary vessels,
		description={tiny blood vessels between the arterials and venules that permit transfer of oxygen, carbon dioxide, nutrients, and waste between body tissues and the blood}
	}
	
	\newglossaryentry{cardiac_muscle}
	{
		name=cardiac muscle,
		description={the heart muscle}
	}

	\newglossaryentry{cardiac_output}
	{
		name=\acrfull{co},
		description={the measure of the volume of blood circulated by the heart in 1 minute; calculated by multiplying the stroke volume by the heart rate}
	}

	\newglossaryentry{carotid_artery}
	{
		name=carotid artery,
		description={the major artery that supplies blood to the head and brain}
	}

	\newglossaryentry{cartilage}
	{
		name=cartilage,
		description={the smooth connective tissue that forms the support structure of the skeletal system and provides cushioning between bones; also forms the nasal septum and portions of the outer ear}
	}

	\newglossaryentry{cecum}
	{
		name=cecum,
		description={the first part of the large intestine, into which the ileum opens}
	}

	\newglossaryentry{central_nervous_system}
	{
		name=\acrfull{cns},
		description={division of \acrlong{pns}; regulates involuntary activities of the body such as heart rate blood pressure and digestion of food}
	}

	\newglossaryentry{cerebellum}
	{
		name=cerebellum,
		description={one of the three major subdivisions of the brain, sometimes called the 'little brain'; coordinates the various activities of the brain, particularly fine body movements}
	}

	\newglossaryentry{cerebrospinal_fluid}
	{
		name=\acrlong{csf},
		description={fluid produced in the ventricles of the brain that flows in the subarachnoid space and bathes the meninges}
	}

	\newglossaryentry{cerebrum}
	{
		name=cerebrum,
		description={the largest part of the three subdivisions of the brain, sometimes called the gray matter; made up of several lobes that control movement, hearing, balance, speech, visual perception, emotions, and personality}
	}

	\newglossaryentry{cervical_spine}
	{
		name=cervical spine,
		description={the portion of the spinal column consisting of the first seven (7) vertebrae that lie in the neck}
	}

	\newglossaryentry{chordae_tendineae}
	{
		name=chordae tendineae,
		description={thin bands of fibrous tissue that attach to the valves in the heart and prevent them from inverting}
	}

	\newglossaryentry{chyme}
	{
		name=chyme,
		description={the substance that leaves the stomach; it is a combination of all the eaten foods with added stomach acids}
	}

	\newglossaryentry{circulatory_system}
	{
		name=circulatory system,
		description={the complex arrangement of connected tubes, including the arteries, arterioles, capillaries, venules, and veins, that moves blood, oxygen, nutrients, carbon dioxide, and cellular waste throughout the body}
	}

	\newglossaryentry{clavicle}
	{
		name=clavicle,
		description={the collar bone; it is lateral to the sternum and anterior to the scapula}
	}

	\newglossaryentry{coccyx}
	{
		name=coccyx,
		description={the last three or four (3-4) vertebrae of the spine; the 'tail bone'}
	}

	\newglossaryentry{coronal_plane}
	{
		name=coronal plane,
		description={an imaginary plane where the body is divided into front and back parts}
	}

	\newglossaryentry{cranium}
	{
		name=cranium,
		description={the area of the head above the ears and eyes; the skull; the cranium contains the brain}
	}

	\newglossaryentry{crioid_cartilage}
	{
		name=crioid cartilage,
		description={A tubular structure }
	}
	
	\newglossaryentry{cricothyroid_membrane}
	{
		name=cricothyroid membrane,
		description={A tubular structure }
	}
	
	\newglossaryentry{dead_space}
	{
		name=dead space,
		description={any portion of the airway that does not contain air and cannot participate in gas exchange, such as the trachea and bronchi}
	}

	\newglossaryentry{dermis} 
	{
		name=dermis,
		description={the inner layer of the skin, containing hair follicles, sweat glands, nerve endings, and blood vessels}
	}
	
	\newglossaryentry{diaphragm}
	{
		name=diaphragm,
		description={muscular dome that forms the undersurface of the thorax, separating the chest from the abdominal cavity. Contraction of this (and the chest wall muscles) brings air into the lungs. Relaxation allows air to be expelled from the lungs}
	}
	
	\newglossaryentry{diastole}
	{
		name=diastole,
		description={relaxation, or period of relaxation, of the heart, especially of the ventricles}
	}
	
	\newglossaryentry{diffusion}
	{
		name=diffusion,
		description={movement of gas from an area of higher concentration to an area of lower concentration}
	}

	\newglossaryentry{digestion}
	{
		name=digestion,
		description={processing of food that nourishes the individual cells of the body}
	}
	
	\newglossaryentry{dorsal_spine}
	{
		name=dorsal spine,
		description={lower part of the back, formed by the lowest five nonfused vertebrae; also called the lumbar spine}
	}
	
	\newglossaryentry{dorsalis_pedis_artery}
	{
		name=dorsalis pedis artery,
		description={artery on the anterior surface of the flow between the first and second metatarsals}
	}

	\newglossaryentry{enteric_nervous_system}
	{
		name=\acrlong{ens},
		description={division of \acrlong{ans}; mesh-like system of neurons that governs the function of the gastrointestinal tract}
	}

	\newglossaryentry{endocrine_system}
	{
		name=endocrine system,
		description={complex message and control system that integrates many of the body's functions, including the release of hormones}
	}
	
	\newglossaryentry{enzyme}
	{
		name=enzyme,
		description={substance designed to speed up the rate of specific biochemical reactions; a biological catalyst}
	}
	
	\newglossaryentry{epinephrine}
	{
		name=epinephrine,
		description={hormone produced by the adrenal medulla that has a vital role in the function of the sympathetic nervous system.  Also called adrenaline}
	}

	\newglossaryentry{epidermis} 
	{
		name=epidermis,
		description={the outer layer of skin that acts as a watertight protective covering}
	}

	\newglossaryentry{epiglottis}
	{
		name=epiglottis,
		description={A tubular structure }
	}
	
	\newglossaryentry{erythrocyte}
	{
		name=erythrocyte,
		description={most common type of red blood cell, whose job is to transport oxygen}
	}
	
	\newglossaryentry{esophagus}
	{
		name=esophagus,
		description={collapsible tube that extends from the pharynx to the stomach; muscle contractions propel food and liquids through it to the stomach}
	}
	
	\newglossaryentry{expiratory_reserve_volume}
	{
		name=expiratory reserve volume,
		description={amount of air that can be exhaled following a normal exhalation; average volume is about 1200 mL in the average adult male}
	}
	
	\newglossaryentry{extension}
	{
		name=extension,
		description={the straightening of a joint}
	}

	\newglossaryentry{fallopian_tubes}
	{
		name=fallopian tubes,
		description={long, slender tubes that extend from the uterus to the region of the ovary on the same side and through which the overpasses from the ovary to the uterus}
	}
	
	\newglossaryentry{femoral_artery}
	{
		name=femoral artery,
		description={the major artery of the thigh, a continuation of the external iliac artery. It supplies blood to lower abdominal wall, external genitalia, and legs. It can be palpated to the groin area}
	}
	
	\newglossaryentry{femoral_head}
	{
		name=femoral head,
		description={proximal end of the femur, articulating with the acetabulum to form the hip joint}
	}
	
	\newglossaryentry{femur}
	{
		name=femur,
		description={the longest and one of the strongest bones in the body.  Also called the thighbone.}
	}
	
	\newglossaryentry{flexion}
	{
		name=flexion,
		description={bending of a joint}
	}
	
	\newglossaryentry{foramen_magnum}
	{
		name=foramen magnum,
		description={large opening at the base of the skull through which the brain connects to the spinal cord}
	}
	
	\newglossaryentry{frontal_bone}
	{
		name=frontal bone,
		description={portion of the cranium that forms the four head}
	}

	\newglossaryentry{gallbladder}
	{
		name=gallbladder,
		description={a sac on the under surface of the liver that collects bile from the liver and discharges it into the duodenum through the common bile duct}
	}
	
	\newglossaryentry{genital_system}
	{
		name=genital system,
		description={reproductive system in men and women}
	}
	
	\newglossaryentry{germinal_layer}
	{
		name=germinal layer,
		description={deepest layer of the epidermis were new skin cells are formed}
	}
	
	\newglossaryentry{greater_trochanter}
	{
		name=greater trochanter,
		description={bony prominence on the proximal lateral side of the thigh, just below the hip joint}
	}
	
	\newglossaryentry{hair_follicles}
	{
		name=hair follicles,
		description={small organs that produce hair}
	}

	\newglossaryentry{heart}
	{
		name=heart,
		description={hollow muscular organ that pumps blood throughout the body}
	}

	\newglossaryentry{heart_rate}
	{
		name=\acrlong{hr},
		description={number of heartbeats during a specific time (usually 1 minute)}
	}

	\newglossaryentry{hinge_joint}
	{
		name=hinge joint,
		description={joint that can bend and straighten but cannot rotate; restricted to motion in one plane}
	}
	
	\newglossaryentry{hormone}
	{
		name=hormone,
		description={substance formed in specialized organs or glands and carried to another organ or group of cells in the same organism; they regulate many body functions, including metabolism, growth, and body temperature}
	}
	
	\newglossaryentry{humerus}
	{
		name=humerus,
		description={supporting bone of the upper arm}
	}
	
	\newglossaryentry{hydrostatic_pressure}
	{
		name=hydrostatic pressure,
		description={pressure water against the walls of its container}
	}
	
	\newglossaryentry{hypoperfusion}
	{
		name=hypoperfusion,
		description={another term for shock}
	}

	\newglossaryentry{hypoxic_drive}
	{
		name=hypoxic drive,
		description={A condition in which chronically low }
	}

	\newglossaryentry{ileum}
	{
		name=ileum,
		description={one of the three bones the fuse to form the pelvic ring}
	}
	
	\newglossaryentry{inferior_vena_cava}
	{
		name=inferior vena cava,
		description={one of the two largest veins in the body; carries blood from the lower extremities and the pelvic and the abdominal organs to the heart}
	}
	
	\newglossaryentry{inspiratory_reserve_volume}
	{
		name=inspiratory reserve volume,
		description={amount of air that can be inhaled after normal inhalation; the amount of air that can be inhaled in addition to the normal title volume}
	}

	\newglossaryentry{intercostal_muscles}
	{
		name=intercostal muscles,
		description={the secondary muscles of respiration.  They include the neck muscles (sternocleidomastoids), the chest pecoralis major muscles, and the abdominal muscles}
	}
		
	\newglossaryentry{interstitial_space}
	{
		name=interstitial space,
		description={space in between the cells}
	}
	
	\newglossaryentry{involuntary_muscle}
	{
		name=involuntary muscle,
		description={muscle over which a person has no conscious control.  It is found in many automatic regulating systems of the body}
	}
	
	\newglossaryentry{ischium}
	{
		name=ischium,
		description={one of three bones that fuse to form the pelvic ring}
	}

	\newglossaryentry{joint}
	{
		name=joint (articulation),
		description={place were two bones come into contact}
	}
	
	\newglossaryentry{joint_capsule}
	{
		name=joint capsule,
		description={fibrous sac that encloses a joint}
	}
	
	\newglossaryentry{kidneys}
	{
		name=kidneys,
		description={two retroperitoneal organs that excrete the end products of metabolism is urine and regulate the body salt and water content}
	}

	\newglossaryentry{labored breathing}
	{
		name=labored breathing,
		description={use of muscles of the chest, back, and abdomen to assist in expanding the chest; occurs when air movement is impaired}
	}
	
	\newglossaryentry{lactic acid}
	{
		name=lactic acid,
		description={a metabolic byproduct of the breakdown of glucose that accumulates when Metabolism proceeds in the absence of oxygen (anaerobic metabolism)}
	}
	
	\newglossaryentry{large intestine}
	{
		name=large intestine,
		description={portion of the digestive to betting circles the abdomen around the small bowel, consisting of the cecum, the colon, and the rectum. It helps regulate water balance and eliminate solid waste}
	}
		
	\newglossaryentry{laryngopharynx}
	{
		name=laryngopharynx,
		description={the secondary muscles of respiration.  They include the neck muscles (sternocleidomastoids), the chest pecoralis major muscles, and the abdominal muscles}
	}

	\newglossaryentry{lesser_trochanter}
	{
		name=lesser trochanter,
		description={projection on the medial superior portion of the femur}
	}
	
	\newglossaryentry{ligament}
	{
		name=ligament,
		description={band of fibrous tissue that connects bones the bones. It supports and strengthens a joint}
	}
	
	\newglossaryentry{liver}
	{
		name=liver,
		description={a large solid organ that lies in the right upper quadrant immediately below the diaphragm; it produces bile, stores glucose for immediate use by the body, and produces many substances that help regulate immune responses}
	}
	
	\newglossaryentry{lumbar_spine}
	{
		name=lumbar spine,
		description={lower part of the back formed by the lowest five nonfused vertebrae; also called the dorsal spine}
	}
	
	\newglossaryentry{lymph}
	{
		name=lymph,
		description={fainting, straw colored fluid that carries oxygen, nutrients, and hormones to the cells and carries waste products of metabolism away from the cells and back into the capillary so that they may be excreted}
	}
	
	\newglossaryentry{lymph_nodes}
	{
		name=lymph nodes,
		description={tiny, oval-shaped structures located in various places along the length vessels that filter lymph}
	}

	\newglossaryentry{mandible}
	{
		name=mandible,
		description={bone of the lower jaw}
	}
	
	\newglossaryentry{menubrium}
	{
		name=menubrium,
		description={upper quarter of the sternum}
	}
	
	\newglossaryentry{maxillae}
	{
		name=maxillae,
		description={upper jaw bones that assist in the formation of the orbit, the nasal cavity, and the pallet and hold the upper teeth}
	}
	
	\newglossaryentry{medulla_oblongata}
	{
		name=medulla oblongata,
		description={nerve tissue that is continuous inferior way with the spinal cord; serves as a conduction pathway for sending and descending nerve tracts; coordinates the heart rate blood vessel diameter, breathing, swallowing, vomiting, coughing, and sneezing}
	}
	
	\newglossaryentry{metabolism}
	{
		name=metabolism,
		description={biochemical processes that result in production of energy from nutrients within cells}
	}
	
	\newglossaryentry{midbrain}
	{
		name=midbrain,
		description={part of the brain that is responsible for helping to regulate the level of consciousness}
	}
	
	\newglossaryentry{midsagittal_plane_(midline)}
	{
		name=midsagittal plane (midline),
		description={imaginary vertical line drawn from the middle of the forehead through the nose and the umbilicus (navel) to the floor, dividing the body and equal left and right halves}
	}
	
	\newglossaryentry{minute_ventilation}
	{
		name=minute ventilation,
		description={see: minute volume}
	}
	
	\newglossaryentry{minute_volume}
	{
		name=minute volume,
		description={volume of air that moved and out of the lungs per minute; calculated by multiplying the title volume and respiratory rate; also called minute ventilation}
	}
	
	\newglossaryentry{motor_nerves}
	{
		name=motor nerves,
		description={nerves that carry information from the central nervous system to the muscles of the body}
	}
	
	\newglossaryentry{mucous membranes}
	{
		name=mucous membranes,
		description={whining of body cavities and passages that communicate directly or indirectly with the environment outside of the body}
	}
	
	\newglossaryentry{mucus}
	{
		name=mucus,
		description={moderate secretion of the mucous membranes that lubricates the body openings}
	}
	
	\newglossaryentry{musculoskeletal system}
	{
		name=musculoskeletal system,
		description={bones involuntary muscles of the body}
	}
	
	\newglossaryentry{myocardium}
	{
		name=myocardium,
		description={heart muscle}
	}
	
	\newglossaryentry{nasopharynx}
	{
		name=nasopharynx,
		description={part of the pharynx that lies above the level of the roof of the mouth, or palate}
	}
	
	\newglossaryentry{nervous system}
	{
		name=nervous system,
		description={system that controls virtually all activities of the body, both voluntary and involuntary}
	}
	
	\newglossaryentry{norepinephrine}
	{
		name=norepinephrine,
		description={neurotransmitter and drug sometimes used in the treatment of shock; produces vasoconstriction to its alpha-stimulator properties}
	}
	
	\newglossaryentry{occiput}
	{
		name=occiput,
		description={most posterior portion of the cranium}
	}
	
	\newglossaryentry{oncotic_pressure}
	{
		name=oncotic pressure,
		description={pressure of water to move, typically into the capillary, as the result of the presence of plasma proteins}
	}
	
	\newglossaryentry{orbit}
	{
		name=orbit,
		description={eye socket, made up of maxilla and zygoma}
	}
	
	\newglossaryentry{oropharynx}
	{
		name=oropharynx,
		description={tubular structure that extends vertically from the back of the mouth to the esophagus and trachea}
	}
	
	\newglossaryentry{ovaries}
	{
		name=ovaries,
		description={female glands that produce sex hormones and (ova)}
	}
	
	\newglossaryentry{palate}
	{
		name=palate,
		description={the "roof" of the mouth}
	}
	
	\newglossaryentry{pancreas}
	{
		name=pancreas,
		description={a flat, solid organ that lies below the liver and the stomach; it is a major source of digestive enzymes that produces the hormone insulin}
	}
	
	\newglossaryentry{parasympathetic_nervous_system}
	{
		name=parasympathetic nervous system,
		description={subdivision of the autonomic nervous system, involved in control of involuntary functions mediated largely by the vagus nerve to the chemical acetylcholine}
	}
	
	\newglossaryentry{parietal_bones}
	{
		name=parietal bones,
		description={bones that lie between the temp oral and occipital regions of the cranium}
	}
	
	\newglossaryentry{patella}
	{
		name=patella,
		description={knee cap; a specialized bone that lies within the tendon of the quadriceps muscle}
	}
	
	\newglossaryentry{pathophysiology}
	{
		name=pathophysiology,
		description={study of how normal physiologic processes are affected by disease}
	}
	
	\newglossaryentry{perfusion}
	{
		name=perfusion,
		description={circulation of oxygenated blood within an organ or tissue in adequate amounts to meet the cells' current needs}
	}
	
	\newglossaryentry{peristalsis}
	{
		name=peristalsis,
		description={the wavelike contraction of the smooth muscle by which the ureters or other tubular organs propelled their contents}
	}
	
	\newglossaryentry{plasma}
	{
		name=plasma,
		description={a sticky, yellow fluid that carries the blood cells and nutrients and transport cellular waste material to the organs of excretion }
	}
	
	\newglossaryentry{platelets}
	{
		name=platelets,
		description={tiny, disc-shaped elements that are much smaller than the cells; they are essential in the initial formation of a blood clot, the mechanism that stops bleeding}
	}
	
	\newglossaryentry{pleura}
	{
		name=pleura,
		description={the Saras membranes covering the lungs and lining the thorax completely enclosing a potential space known as the pleural space}
	}
	
	\newglossaryentry{pleural_space}
	{
		name=pleural space,
		description={potential space between the parietal pleura and of the visceral pleura; described as "potential" because under normal conditions, the spaces not exist}
	}
	
	\newglossaryentry{pons}
	{
		name=pons,
		description={organ that lies below the midbrain and above the medulla and contains numerous important nerve fibers, including those for sleep, respiration, and the medullary respiratory center}
	}
	
	\newglossaryentry{posterior_tibial_artery}
	{
		name=posterior tibial artery,
		description={artery just behind the medial malleolus; supplies blood to the foot}
	}
	
	\newglossaryentry{prostate_gland}
	{
		name=prostate gland,
		description={small gland that surrounds the male urethra where it emerges from the urinary bladder; it's increase the fluid that is part of the ejaculatory fluid}
	}
	
	\newglossaryentry{pubic_symphysis}
	{
		name=pubic symphysis,
		description={hard, bony, and cartilaginous prominence found at the midline in the lowermost portion of the abdomen where the two halves of the pelvic ring are joined by cartilage at a joint with minimal motion}
	}
	
	\newglossaryentry{pubis}
	{
		name=pubis,
		description={one of the three bones that fuse to form the pelvic ring}
	}
	
	\newglossaryentry{pulmonary_artery}
	{
		name=pulmonary artery,
		description={the major artery leading from the right ventricle of the heart to the lungs; carries oxygen-poor blood}
	}
	
	\newglossaryentry{pulmonary_circulation}
	{
		name=pulmonary circulation,
		description={flow of blood from the right ventricle through the pulmonary arteries and all of their branches and capillaries in the lungs and back to the left atrium through the venules and pulmonary veins; also called the lesser circulation}
	}
	
	\newglossaryentry{pulmonary_veins}
	{
		name=pulmonary veins,
		description={four veins that return oxygenated blood from the lungs to the left atrium of the heart}
	}
	
	\newglossaryentry{pulse}
	{
		name=pulse,
		description={wave of pressure created as the heart contracts of horses blood out of the left ventricle and into the major arteries}
	}
	
	\newglossaryentry{radial_artery}
	{
		name=radial artery,
		description={major artery in the forearm; it is palpable at the wrist on the thumb side}
	}
	
	\newglossaryentry{radius}
	{
		name=radius,
		description={the bone on the thumb side of the forearm}
	}
	
	\newglossaryentry{rectum}
	{
		name=rectum,
		description={the lowermost end of the:}
	}
	
	\newglossaryentry{red_blood_cell}
	{
		name=red blood cell,
		description={cell that carries oxygen to the body's tissues; also an called erythrocyte}
	}
	
	\newglossaryentry{renal_pelvis}
	{
		name=Renal pelvis,
		description={cone-shaped area that collects urine from the kidneys and funnels it through the ureter into the bladder}
	}
	
	\newglossaryentry{residual_volume}
	{
		name=residual volume,
		description={air that remains in the lungs after maximal expiration}
	}
	
	\newglossaryentry{respiration}
	{
		name=respiration,
		description={inhaling and exhaling of air; the physiologic process that exchanges carbon dioxide from fresh air}
	}
	
	\newglossaryentry{respiratory_compromise}
	{
		name=respiratory compromise,
		description={inability of the body to move gas effectively}
	}
	
	\newglossaryentry{respiratory_system}
	{
		name=respiratory system,
		description={all the structures of the body that contribute to the process of breathing, consisting of the upper and lower airways and their component parts}
	}
	
	\newglossaryentry{reticular_activating_system}
	{
		name=reticular activating system,
		description={located in the upper brainstem; responsible for the maintenance of consciousness, specifically one's level of arousal}
	}
	
	\newglossaryentry{retroperitoneal}
	{
		name=retroperitoneal,
		description={behind the abdominal cavity}
	}
	
	\newglossaryentry{sacroiliac_joint}
	{
		name=sacroiliac joint,
		description={connection point between the pelvis and the vertebral column}
	}
	
	\newglossaryentry{sacrum}
	{
		name=sacrum,
		description={one of the three bones (sacrum and two pelvic bones) that make up the pelvic ring; consists of five fused sacral vertebrae}
	}
	
	\newglossaryentry{sagittal_(lateral)_plane}
	{
		name=sagittal (lateral) plane,
		description={imaginary line where the body is divided into left and right parts}
	}
	
	\newglossaryentry{salivary_glands}
	{
		name=salivary glands,
		description={glands that produce saliva to keep the mouth and pharynx moist}
	}
	
	\newglossaryentry{scalp}
	{
		name=scalp,
		description={thick skin covering the cranium which usually bears hair}
	}
	
	\newglossaryentry{scapula}
	{
		name=scapula,
		description={the shoulder blade}
	}
	
	\newglossaryentry{sebaceous_glands}
	{
		name=sebaceous glands,
		description={glands that produce an oily substance called sebum, which discharges along the shafts of the hairs}
	}
	
	\newglossaryentry{semen}
	{
		name=semen,
		description={fluid ejaculated from the penis and containing sperm}
	}
	
	\newglossaryentry{seminal_vesicles}
	{
		name=seminal vesicles,
		description={storage sacs for sperm and seminal fluid which empty into the urethra at the prostate}
	}
	
	\newglossaryentry{sensory_nerves}
	{
		name=sensory nerves,
		description={nerves that carry sensations such as  touch, taste, smell, heat, cold, and pain from the body to the central nervous system}
	}
	
	\newglossaryentry{shock}
	{
		name=shock,
		description={abnormal state associated with the inadequate oxygen and nutrient delivery to the cells of the body, also known as hypoperfusion}
	}
	
	\newglossaryentry{shoulder_girdle}
	{
		name=shoulder girdle,
		description={the proximal portion of the upper extremities, made up of the clavicle, the scapula, and the humerus}
	}
	
	\newglossaryentry{skeletal_muscle}
	{
		name=skeletal muscle,
		description={muscle that is attached to bones and usually crosses at least one joint; striated, or voluntary, muscle}
	}
	
	\newglossaryentry{skeleton}
	{
		name=skeleton,
		description={framework that gives the body its recognizable form; also designed to allow motion of the body and protection of vital organs}
	}
	
	\newglossaryentry{small_intestine}
	{
		name=small intestine,
		description={portion of the digestive to between the stomach and the cecum, consisting of the duodenum, jejunum, and ileum}
	}
	
	\newglossaryentry{smooth_muscle}
	{
		name=smooth muscle,
		description={involuntary muscle; it constitutes the bulk of the gastrointestinal tract and is present in nearly every organ to regulate automatic activity}
	}
	
	\newglossaryentry{somatic_nervous_system}
	{
		name=somatic nervous system,
		description={part of the nervous system that regulates activities over which there is voluntary control}
	}
	
	\newglossaryentry{sphincter}
	{
		name=sphincter,
		description={muscle arranged in  a circle that is able to decrease the diameter of tubes. Examples are found within the rectum, bladder, and blood vessels}
	}
	
	\newglossaryentry{sphygmomanometer}
	{
		name=sphygmomanometer,
		description={device used to measure blood pressure}
	}
	
	\newglossaryentry{spinal_cord}
	{
		name=spinal cord,
		description={extension of the brain, composed of virtually all the nerves carry messages between the brain and the rest of the body. It lies inside of and is protected by the spinal canal}
	}
	
	\newglossaryentry{sternum}
	{
		name=sternum,
		description={breastbone}
	}
	
	\newglossaryentry{stratum_corneal_layer}
	{
		name=stratum corneal layer,
		description={outermost board dead layer of the skin}
	}
	
	\newglossaryentry{stroke volume}
	{
		name=\acrlong{sv},
		description={volume of blood pumped forward with each ventricular contraction}
	}
	
	\newglossaryentry{subcutaneous_tissue}
	{
		name=subcutaneous tissue,
		description={tissue, largely fat, that lies directly under the dermis and serves as an insulator of the body}
	}
	
	\newglossaryentry{superior_vena_cava}
	{
		name=superior vena cava,
		description={one of the two largest veins in the body; carries blood from the upper extremities, head, neck, and chest into the heart}
	}
	
	\newglossaryentry{sweat_glands}
	{
		name=sweat glands,
		description={glands that secrete sweat located in the dermal layer of the skin}
	}
	
	\newglossaryentry{symphysis}
	{
		name=symphysis,
		description={type of joint that is grown together to form a very stable connection}
	}
	
	\newglossaryentry{synovial_fluid}
	{
		name=synovial fluid,
		description={small amount of liquid within a joint use as lubrication}
	}
	
	\newglossaryentry{synovial_membrane}
	{
		name=synovial membrane,
		description={lighting of a joint that secrete synovial fluid into the joint space}
	}
	
	\newglossaryentry{systemic_circulation}
	{
		name=systemic circulation,
		description={portion of the circulatory system outside of the heart and lungs}
	}
	
	\newglossaryentry{systemic_vascular_resistance}
	{
		name=\acrlong{svr},
		description={resistance that blood must overcome to be able to move within the blood vessels; related to the amount of dilation or constriction in the blood vessel}
	}

	\newglossaryentry{systole}
	{
		name=systole,
		description={contraction, or period of contraction, of the heart, especially that of the ventricles}
	}
	
	\newglossaryentry{temporal_bones}
	{
		name=temporal bones,
		description={lateral bones on each side of the cranium; the temples}
	}
	
	\newglossaryentry{tendons}
	{
		name=tendons,
		description={fibrous connective tissue that attaches muscle to bone}
	}
	
	\newglossaryentry{testicle}
	{
		name=testicle,
		description={it male genital land that contain specialized cells that produce hormones and sperm}
	}
	
	\newglossaryentry{thoracic_cage}
	{
		name=thoracic cage,
		description={chest or rib cage}
	}
	
	\newglossaryentry{thoracic_spine}
	{
		name=thoracic spine,
		description={12 vertebrae that lie between the cervical vertebrae and the lumbar vertebrae. One pair of ribs is attached to each of these vertebrae}
	}
	
	\newglossaryentry{thorax}
	{
		name=thorax,
		description={chest cavity contains the heart, lungs, esophagus, and great vessels}
	}
	
	\newglossaryentry{thighbone}
	{
		name=thighbone,
		description={another name for the femur}
	}
	
	\newglossaryentry{thyroid_cartilage}
	{
		name=thyroid cartilage,
		description={firm prominence of cartilage that forms the upper part of the larynx; the Adam's apple}
	}
	
	\newglossaryentry{tibia}
	{
		name=tibia,
		description={shinbone; larger of the two bones of the lower leg}
	}
	
	\newglossaryentry{tidal_volume}
	{
		name=tidal volume,
		description={amount of air moved in and out of the lungs are one relaxed breath; about 500 mL for an adult}
	}
	
	\newglossaryentry{topographic_anatomy}
	{
		name=topographic anatomy,
		description={the superficial landmarks of the body that serve as guides to the structures that lie beneath them}
	}
	
	\newglossaryentry{trachea}
	{
		name=trachea,
		description={the windpipe; main trunk for air passing to and from the lungs}
	}
	
	\newglossaryentry{transverse_plane}
	{
		name=transverse plane,
		description={an imaginary line with the body is divided in the top and bottom parts.  Also known as the \gls{axial_plane}}
	}
	
	\newglossaryentry{triceps}
	{
		name=triceps,
		description={muscle in the back of the upper arm}
	}
	
	\newglossaryentry{tunica media}
	{
		name=tunica media,
		description={middle and thickest part of tissue of a blood vessel wall, composed of elastic tissue and smooth muscle cells that allow the vessel to expand or contract in response to changes in blood pressure and tissue demand}
	}
	
	\newglossaryentry{ulna}
	{
		name=ulna,
		description={enter bone of the forearm, on the side opposite the thumb}
	}
	
	\newglossaryentry{ureter}
	{
		name=ureter,
		description={small, hollow tube that carries urine from the kidneys to the bladder}
	}
	
	\newglossaryentry{urethra}
	{
		name=urethra,
		description={canal that conveys urine from the bladder to the outside of the body}
	}
	
	\newglossaryentry{urinary_bladder}
	{
		name=urinary bladder,
		description={a sac behind the pubic symphysis made of smooth muscle that collects and stores urine}
	}
	
	\newglossaryentry{urinary_system}
	{
		name=urinary system,
		description={organs that control the discharge of certain waste materials filtered from the blood and excreted as urine}
	}
	
	\newglossaryentry{vagina}
	{
		name=vagina,
		description={muscular, dispensable to that connects the uterus with the vulva (the external female genitalia); also called the birth canal}
	}
	
	\newglossaryentry{vasa_deferentia}
	{
		name=vasa deferentia,
		description={spermatic duct of the testicles; also called the vas deferens}
	}
	
	\newglossaryentry{vas_deferens}
	{
		name=vas deferens,
		description={see: vasa deferentia}
	}
	
	\newglossaryentry{ventilation}
	{
		name=ventilation,
		description={movement of air between the lungs and the environment}
	}
	
	\newglossaryentry{ventricle}
	{
		name=ventricle,
		description={one of two lower chambers of the heart}
	}
	
	\newglossaryentry{vertebrae}
	{
		name=vertebrae,
		description={the 33 bones that make up the spinal column}
	}
	
	\newglossaryentry{voluntary_muscle}
	{
		name=voluntary muscle,
		description={muscle that is under direct voluntary control of the brain can be contracted or relax that will; skeletal, or striated, muscle}
	}
	
	\newglossaryentry{vq_ratio}
	{
		name=V/Q ratio,
		description={measurement that examines how much gas is being moved effectively and how much blood is flowing around the alveoli or gas exchange (perfusion) occurs}
	}
	
	\newglossaryentry{white_blood_cell}
	{
		name=white blood cell,
		description={blood cell that has a role in the body's immune defense against infection; also called a leukocyte}
	}
	
	\newglossaryentry{xiphoid_process}
	{
		name=xiphoid process,
		description={narrow, cartilaginous lower tip of the sternum}
	}
	
	\newglossaryentry{zygomas}
	{
		name=zygomas,
		description={the quadrangualar bones of the cheek, articulating with the frontal bone, the maxillae, the zygomatic processes of the temporal bone, and the great wings of the sphenoid bone}
	}

\chapter{The Human Body}

\subsection*{Abbreviations}
\begin{description}[leftmargin=!,labelwidth=\widthof{\bfseries ABCDF}]
	\item [\acrshort{ans}] 			\acrlong{ans}
	\item [\acrshort{co}] 			\acrlong{co}
	\item [\acrshort{cns}] 			\acrlong{cns}
	\item [\acrshort{ens}] 			\acrlong{ens}
	\item [\acrshort{hr}] 			\acrlong{hr}
	\item [\acrshort{pns}] 			\acrlong{pns}
	\item [\acrshort{psns}] 		\acrlong{psns}
	\item [\acrshort{sns}] 			\acrlong{sns}
	\item [\acrshort{sv}] 			\acrlong{sv}
\end{description}\hfill \\

\subsection*{Definitions}
\begin{description}
	\item [\gls{abdomen}] 							\glsdesc{abdomen}
	\item [\gls{accessory_muscles}] 				\glsdesc{accessory_muscles}
	\item [\gls{acetabulum}] 						\glsdesc{acetabulum}
	\item [\gls{adams_apple}] 						\glsdesc{adams_apple}
	\item [\gls{adenosine_triphosphate}] 			\glsdesc{adenosine_triphosphate}
	\item [\gls{adrenal_gland}] 					\glsdesc{adrenal_gland}
	\item [\gls{adrenaline}] 						\glsdesc{adrenaline}
	\item [\gls{adrenergic}] 						\glsdesc{adrenergic}
	\item [\gls{aerobic_metabolism}] 				\glsdesc{aerobic_metabolism}
	\item [\gls{agonal_gasps}] 						\glsdesc{agonal_gasps}
	\item [\gls{alpha_adrenergic_receptors}] 		\glsdesc{alpha_adrenergic_receptors}
	\item [\gls{alveoli}] 							\glsdesc{alveoli}
	\item [\gls{anaerobic_metabolism}] 				\glsdesc{anaerobic_metabolism}
	\item [\gls{anatomic_position}] 				\glsdesc{anatomic_position}
	\item [\gls{aorta}]								\glsdesc{aorta}
	\item [\gls{appendicular_skeleton}] 			\glsdesc{appendicular_skeleton}
	\item [\gls{appendix}] 							\glsdesc{appendix}
	\item [\gls{arteriole}] 						\glsdesc{arteriole}
	\item [\gls{artery}] 							\glsdesc{artery}
	\item [\gls{articular_cartilage}] 				\glsdesc{articular_cartilage}
	\item [\gls{atrium}] 							\glsdesc{atrium}
	\item [\gls{autonomic_nervous_system}] 			\glsdesc{autonomic_nervous_system}
	\item [\gls{axial_skeleton}] 					\glsdesc{axial_skeleton}
	\item [\gls{ball_and_socket_joint}]				\glsdesc{ball_and_socket_joint}
	\item [\gls{beta_adrenergic_receptors}]			\glsdesc{beta_adrenergic_receptors}
	\item [\gls{biceps}]							\glsdesc{biceps}
	\item [\gls{bile_duct}]							\glsdesc{bile_duct}
	\item [\gls{blood_pressure}]					\glsdesc{blood_pressure}
	\item [\gls{brachial_artery}]					\glsdesc{brachial_artery}
	\item [\gls{brain}]								\glsdesc{brain}
	\item [\gls{brainstem}]							\glsdesc{brainstem}
	
	\item [\gls{capillary}] 						\glsdesc{capillary}
	\item [\gls{capillary_vessels}] 				\glsdesc{capillary_vessels}
	\item [\gls{cardiac_muscle}] 					\glsdesc{cardiac_muscle}
	
	\item [\gls{cardiac_output}] 					\glsdesc{cardiac_output}
	\item [\gls{carotid_artery}] 					\glsdesc{carotid_artery}
	\item [\gls{cartilage}] 						\glsdesc{cartilage}
	\item [\gls{cecum}] 							\glsdesc{cecum}
	\item [\gls{central_nervous_system}] 			\glsdesc{central_nervous_system}
	\item [\gls{cerebellum}] 						\glsdesc{cerebellum}
	\item [\gls{cerebrospinal_fluid}] 				\glsdesc{cerebrospinal_fluid}
	\item [\gls{cerebrum}] 							\glsdesc{cerebrum}
	\item [\gls{cervical_spine}] 					\glsdesc{cervical_spine}
	\item [\gls{chordae_tendineae}] 				\glsdesc{chordae_tendineae}
	\item [\gls{chyme}] 							\glsdesc{chyme}
	\item [\gls{circulatory_system}] 				\glsdesc{circulatory_system}
	\item [\gls{clavicle}] 							\glsdesc{clavicle}
	\item [\gls{coccyx}] 							\glsdesc{coccyx}
	\item [\gls{coronal_plane}] 					\glsdesc{coronal_plane}
	\item [\gls{cranium}] 							\glsdesc{cranium}
	\item [\gls{crioid_cartilage}] 					\glsdesc{crioid_cartilage}
	\item [\gls{cricothyroid_membrane}] 			\glsdesc{cricothyroid_membrane}
	\item [\gls{dead_space}]	 					\glsdesc{dead_space}
	\item [\gls{dermis}]	 						\glsdesc{dermis}
	\item [\gls{diaphragm}] 						\glsdesc{diaphragm}
	\item [\gls{diastole}] 							\glsdesc{diastole}
	\item [\gls{diffusion}] 						\glsdesc{diffusion}
	\item [\gls{digestion}] 						\glsdesc{digestion}
	\item [\gls{dorsalis_pedis_artery}] 			\glsdesc{dorsalis_pedis_artery}
	\item [\gls{endocrine_system}] 					\glsdesc{endocrine_system}
	\item [\gls{enzyme}] 							\glsdesc{enzyme}
	\item [\gls{epidermis}]	 						\glsdesc{epidermis}
	\item [\gls{epiglottis}]	 					\glsdesc{epiglottis}
	\item [\gls{epinephrine}]	 					\glsdesc{epinephrine}
	\item [\gls{esophagus}]	 						\glsdesc{esophagus}
	\item [\gls{expiratory_reserve_volume}]	 		\glsdesc{expiratory_reserve_volume}
	\item [\gls{extension}]	 						\glsdesc{extension}
	\item [\gls{fallopian_tubes}]                   \glsdesc{fallopian_tubes}
	\item [\gls{femoral_artery}]                    \glsdesc{femoral_artery}
	\item [\gls{femoral_head}]                      \glsdesc{femoral_head}
	\item [\gls{femur}]                             \glsdesc{femur}
	\item [\gls{flexion}]                           \glsdesc{flexion}
	\item [\gls{foramen_magnum}]                    \glsdesc{foramen_magnum}
	\item [\gls{frontal_bone}]                      \glsdesc{frontal_bone}
	\item [\gls{gallbladder}]                       \glsdesc{gallbladder}
	\item [\gls{genital_system}]                    \glsdesc{genital_system}
	\item [\gls{germinal_layer}]                    \glsdesc{germinal_layer}
	\item [\gls{greater_trochanter}]                \glsdesc{greater_trochanter}
	\item [\gls{hair_follicles}]                    \glsdesc{hair_follicles}
	\item [\gls{heart}]                             \glsdesc{heart}
	\item [\gls{heart_rate}]                        \glsdesc{heart_rate}
	\item [\gls{hinge_joint}]						\glsdesc{hinge_joint}
	\item [\gls{hormone}]                           \glsdesc{hormone}
	\item [\gls{humerus}]                           \glsdesc{humerus}
	\item [\gls{hydrostatic_pressure}]              \glsdesc{hydrostatic_pressure}
	\item [\gls{hypoperfusion}]                     \glsdesc{hypoperfusion}
	\item [\gls{hypoxic_drive}]                     \glsdesc{hypoxic_drive}
	\item [\gls{ileum}]                             \glsdesc{ileum}
	\item [\gls{inferior_vena_cava}]                \glsdesc{inferior_vena_cava}
	\item [\gls{inspiratory_reserve_volume}]        \glsdesc{inspiratory_reserve_volume}
	\item [\gls{intercostal_muscles}]               \glsdesc{intercostal_muscles}
	\item [\gls{interstitial_space}]                \glsdesc{interstitial_space}
	\item [\gls{involuntary_muscle}]                \glsdesc{involuntary_muscle}
	\item [\gls{ischium}]                           \glsdesc{ischium}
	\item [\gls{joint}]                             \glsdesc{joint}
	\item [\gls{joint_capsule}]                     \glsdesc{joint_capsule}
	\item [\gls{kidneys}]                           \glsdesc{kidneys}
	\item [\gls{labored breathing}]                 \glsdesc{labored breathing}
	\item [\gls{lactic acid}]                       \glsdesc{lactic acid}
	\item [\gls{large intestine}]                   \glsdesc{large intestine}
	\item [\gls{laryngopharynx}]                    \glsdesc{laryngopharynx}
	\item [\gls{lesser_trochanter}]                 \glsdesc{lesser_trochanter}
	\item [\gls{ligament}]                          \glsdesc{ligament}
	\item [\gls{liver}]                             \glsdesc{liver}
	\item [\gls{lumbar_spine}]                      \glsdesc{lumbar_spine}
	\item [\gls{lymph}]                             \glsdesc{lymph}
	\item [\gls{lymph_nodes}]                       \glsdesc{lymph_nodes}
	\item [\gls{mandible}]                          \glsdesc{mandible}
	\item [\gls{menubrium}]                         \glsdesc{menubrium}
	\item [\gls{maxillae}]                          \glsdesc{maxillae}
	\item [\gls{medulla_oblongata}]                 \glsdesc{medulla_oblongata}
	\item [\gls{metabolism}]                        \glsdesc{metabolism}
	\item [\gls{midbrain}]                          \glsdesc{midbrain}
	\item [\gls{midsagittal_plane_(midline)}]       \glsdesc{midsagittal_plane_(midline)}
	\item [\gls{minute_ventilation}]                \glsdesc{minute_ventilation}
	\item [\gls{minute_volume}]                     \glsdesc{minute_volume}
	\item [\gls{motor_nerves}]                      \glsdesc{motor_nerves}
	\item [\gls{mucous membranes}]                  \glsdesc{mucous membranes}
	\item [\gls{mucus}]                             \glsdesc{mucus}
	\item [\gls{musculoskeletal system}]            \glsdesc{musculoskeletal system}
	\item [\gls{myocardium}]                        \glsdesc{myocardium}
	\item [\gls{nasopharynx}]                       \glsdesc{nasopharynx}
	\item [\gls{nervous system}]                    \glsdesc{nervous system}
	\item [\gls{norepinephrine}]                    \glsdesc{norepinephrine}
	\item [\gls{occiput}]                           \glsdesc{occiput}
	\item [\gls{oncotic_pressure}]                  \glsdesc{oncotic_pressure}
	\item [\gls{orbit}]                             \glsdesc{orbit}
	\item [\gls{ovaries}]                           \glsdesc{ovaries}
	\item [\gls{palate}]                            \glsdesc{palate}
	\item [\gls{pancreas}]                          \glsdesc{pancreas}
	\item [\gls{parasympathetic_nervous_system}]    \glsdesc{parasympathetic_nervous_system}
	\item [\gls{parietal_bones}]                    \glsdesc{parietal_bones}
	\item [\gls{patella}]                           \glsdesc{patella}
	\item [\gls{pathophysiology}]                   \glsdesc{pathophysiology}
	\item [\gls{perfusion}]                         \glsdesc{perfusion}
	\item [\gls{peristalsis}]                       \glsdesc{peristalsis}
	\item [\gls{plasma}]                            \glsdesc{plasma}
	\item [\gls{pleura}]                            \glsdesc{pleura}
	\item [\gls{pleural_space}]                     \glsdesc{pleural_space}
	\item [\gls{posterior_tibial_artery}]           \glsdesc{posterior_tibial_artery}
	\item [\gls{prostate_gland}]                    \glsdesc{prostate_gland}
	\item [\gls{pubic_symphysis}]                   \glsdesc{pubic_symphysis}
	\item [\gls{pubis}]                             \glsdesc{pubis}
	\item [\gls{pulmonary_artery}]                  \glsdesc{pulmonary_artery}
	\item [\gls{pulmonary_circulation}]             \glsdesc{pulmonary_circulation}
	\item [\gls{pulmonary_veins}]                   \glsdesc{pulmonary_veins}
	\item [\gls{pulse}]                             \glsdesc{pulse}
	\item [\gls{radial_artery}]                     \glsdesc{radial_artery}
	\item [\gls{radius}]                            \glsdesc{radius}
	\item [\gls{rectum}]                            \glsdesc{rectum}
	\item [\gls{red_blood_cell}]                    \glsdesc{red_blood_cell}
	\item [\gls{renal_pelvis}]                      \glsdesc{renal_pelvis}
	\item [\gls{residual_volume}]                   \glsdesc{residual_volume}
	\item [\gls{respiration}]                       \glsdesc{respiration}
	\item [\gls{respiratory_compromise}]            \glsdesc{respiratory_compromise}
	\item [\gls{respiratory_system}]                \glsdesc{respiratory_system}
	\item [\gls{reticular_activating_system}]       \glsdesc{reticular_activating_system}
	\item [\gls{retroperitoneal}]                   \glsdesc{retroperitoneal}
	\item [\gls{sacroiliac_joint}]                  \glsdesc{sacroiliac_joint}
	\item [\gls{sacrum}]                            \glsdesc{sacrum}
	\item [\gls{sagittal_(lateral)_plane}]          \glsdesc{sagittal_(lateral)_plane}
	\item [\gls{salivary_glands}]                   \glsdesc{salivary_glands}
	\item [\gls{scalp}]                             \glsdesc{scalp}
	\item [\gls{scapula}]                           \glsdesc{scapula}
	\item [\gls{sebaceous_glands}]                  \glsdesc{sebaceous_glands}
	\item [\gls{semen}]                             \glsdesc{semen}
	\item [\gls{seminal_vesicles}]                  \glsdesc{seminal_vesicles}
	\item [\gls{sensory_nerves}]                    \glsdesc{sensory_nerves}
	\item [\gls{shock}]                             \glsdesc{shock}
	\item [\gls{shoulder_girdle}]                   \glsdesc{shoulder_girdle}
	\item [\gls{skeletal_muscle}]                   \glsdesc{skeletal_muscle}
	\item [\gls{skeleton}]                          \glsdesc{skeleton}
	\item [\gls{small_intestine}]                   \glsdesc{small_intestine}
	\item [\gls{smooth_muscle}]                     \glsdesc{smooth_muscle}
	\item [\gls{somatic_nervous_system}]            \glsdesc{somatic_nervous_system}
	\item [\gls{sphincter}]                         \glsdesc{sphincter}
	\item [\gls{sphygmomanometer}]                  \glsdesc{sphygmomanometer}
	\item [\gls{spinal_cord}]                       \glsdesc{spinal_cord}
	\item [\gls{sternum}]                           \glsdesc{sternum}
	\item [\gls{stratum_corneal_layer}]             \glsdesc{stratum_corneal_layer}
	\item [\gls{stroke volume}]                     \glsdesc{stroke volume}
	\item [\gls{subcutaneous_tissue}]               \glsdesc{subcutaneous_tissue}
	\item [\gls{superior_vena_cava}]                \glsdesc{superior_vena_cava}
	\item [\gls{sweat_glands}]                      \glsdesc{sweat_glands}
	\item [\gls{symphysis}]                         \glsdesc{symphysis}
	\item [\gls{synovial_fluid}]                    \glsdesc{synovial_fluid}
	\item [\gls{synovial_membrane}]                 \glsdesc{synovial_membrane}
	\item [\gls{systemic_circulation}]              \glsdesc{systemic_circulation}
	\item [\gls{systemic_vascular_resistance}]      \glsdesc{systemic_vascular_resistance}
	\item [\gls{systole}]                           \glsdesc{systole}
	\item [\gls{temporal_bones}]                    \glsdesc{temporal_bones}
	\item [\gls{tendons}]                           \glsdesc{tendons}
	\item [\gls{testicle}]                          \glsdesc{testicle}
	\item [\gls{thoracic_cage}]                     \glsdesc{thoracic_cage}
	\item [\gls{thoracic_spine}]                    \glsdesc{thoracic_spine}
	\item [\gls{thorax}]                            \glsdesc{thorax}
	\item [\gls{thighbone}]                         \glsdesc{thighbone}
	\item [\gls{thyroid_cartilage}]                 \glsdesc{thyroid_cartilage}
	\item [\gls{tibia}]                             \glsdesc{tibia}
	\item [\gls{tidal_volume}]                      \glsdesc{tidal_volume}
	\item [\gls{topographic_anatomy}]               \glsdesc{topographic_anatomy}
	\item [\gls{trachea}]                           \glsdesc{trachea}
	\item [\gls{transverse_plane}]        		  	\glsdesc{transverse_plane}
	\item [\gls{triceps}]                           \glsdesc{triceps}
	\item [\gls{tunica media}]                      \glsdesc{tunica media}
	\item [\gls{ulna}] 								\glsdesc{ulna}
	\item [\gls{ureter}] 							\glsdesc{ureter}
	\item [\gls{urethra}] 							\glsdesc{urethra}
	\item [\gls{urinary_bladder}] 					\glsdesc{urinary_bladder}
	\item [\gls{urinary_system}] 					\glsdesc{urinary_system}
	\item [\gls{vagina}] 							\glsdesc{vagina}
	\item [\gls{vas_deferens}] 						\glsdesc{vas_deferens}
	\item [\gls{ventilation}] 						\glsdesc{ventilation}
	\item [\gls{ventricle}] 						\glsdesc{ventricle}
	\item [\gls{vertebrae}] 						\glsdesc{vertebrae}
	\item [\gls{voluntary_muscle}] 					\glsdesc{voluntary_muscle}
	\item [\gls{vq_ratio}] 							\glsdesc{vq_ratio}
	\item [\gls{white_blood_cell}] 					\glsdesc{white_blood_cell}
	\item [\gls{xiphoid_process}] 					\glsdesc{xiphoid_process}
	\item [\gls{zygomas}] 							\glsdesc{zygomas}
\end{description}

\section{Correct Medical Terminology}

\begin{table}[]
	\caption{Anatomic planes}
	\label{tab:Anatomic_planes}
	\begin{tabular}{|l|l|l|}
		\hline
		divides body into & name       	& also known as 	\\ \hline
		\hline 
		left and right    & sagittal   	& lateral       	\\ \hline
		front and back    & frontal 	& coronal         	\\ \hline
		waist             & transverse 	& axial         	\\ \hline
	\end{tabular}
\end{table}

%\clearpage

%\section{Skeletal}
%Two (2) "sub"-skeletons:
%\paragraph{axial} (head, spine, thoracic cage) and
%\paragraph{appendicular} (arms, leg, pelvis, and their attachment points)

%\subsection{skull}
%Two groups:
%\paragraph{cranium}
%\paragraph{}

%\subsection{Skull}

%\afterpage{%
%\clearpage
%\subsection{Potential Test Questions}
%\begin{outline}[enumerate]
%	\1 Why do we need a template here?
%	\1[] Because LaTeX is stupid so we have to.
%\end{outline}
%}

\clearpage
\end{document}


\begin{document}
\setcounter{chapter}{8}
\label{ch:chapter9}
\clearpage

% Glossary acronym entries %
	\newacronym{avpu}{AVPU}{Alert Verbal Pain Unresponsive}
	\newacronym{co2}{CO_{2}}{carbon dioxide}
	\newacronym{dcapbtls}{DCAP-BTLS}{Deformities, Contusions, Abrasions, Punctures/pentetrations, Burns, Tenderness, Lacerations, Swelling}
	\newacronym{ics}{ICS}{incident command system}
	\newacronym{loc}{LOC}{level of conciousness}
	\newacronym{moi}{MOI}{mechanism of injury}
	\newacronym{noi}{NOI}{nature of illness}
	\newacronym{opqrst}{OPQRST}{Onset, Provocaiton/palliation, Quality, Region/radiation, Severity, Timing}
	\newacronym{opqrstu}{OPQRSTU}{Onset, Provocaiton/palliation, Quality, Region/radiation, Severity, Timing, What Have 'U' Done?}
	\newacronym{sample}{SAMPLE}{TODO}

% Glossary entries
	% accessory_muscles defined in ch01

	\newglossaryentry{altered_mental_status}
	{
		name=altered mental status,
		description={any deviation from alert and oriented to person, place, time, and event, or any deviation from a patient's normal baseline mental status}
	}

	\newglossaryentry{auscultate}
	{
		name=auscultate,
		description={to listen to sounds within an organ with a stethoscpe}
	}

	\newglossaryentry{avpu_scale}
	{
		name=\acrshort{AVPU} scale,
		description={a method of assessing the level of conciousness by determining whether the patient is awake and alert, responsive to verbal stimuli or pain, or unresponsive; used principally early in the assessment process}
	}

	% blood_pressure already defined in ch06
	
	\newglossaryentry{bradycardia}
	{
		name=bradycardia,
		description={ }
	}

	\newglossaryentry{breath_sounds}
	{
		name=breath sounds,
		description={an indication of air movement in the lungs, usually associated with a stethoscope}
	}

	\newglossaryentry{capillary_refill}
	{
		name=capillary refill,
		description={a test that evaluates distal circulatory system function by squeezing (blanching) blood from an area such as a nail bed and wathcing the speed of its return after releasing the pressure}
	}

	\newglossaryentry{capnography}
	{
		name=capnography,
		description={a noninvasive method to quickly and efficiently provide information on a patient's ventilatory status, circulation, and metabolism; effectively measures the concentration of carbon dioxide in expired air over time}
	}

	\newglossaryentry{carbon_dioxide}
	{
		name=\acrfull{co2},
		description={a compoent of air and typically makes up 0.3\% of air at sea level; also a waste product exhaled during expiration the respiratory system}
	}

	\newglossaryentry{chief_complaint}
	{
		name=chief_complaint,
		description={the reason a patient called for help; also, the patient's response to questions such as "What's wrong?" or "What happened?"}
	}

	\newglossaryentry{coagulate}
	{
		name=coagulate,
		description={to form a clot to plug an opening in an injured blood cessel and stop bleeding}
	}

	\newglossaryentry{conjunctiva}
	{
		name=conjunctiva,
		description={the delicate membrane that lines the eyelids and covers the exposed surface of the eye}
	}

	\newglossaryentry{crackles}
	{
		name=crackles,
		description={a crackling, rattling breath sound signals fluid in the air spaces of the lungs}
	}

	\newglossaryentry{crepitus}
	{
		name=crepitus,
		description={a grating or grinding ssensation caused by fractured bone ends or joints rubbing together;
		also air bubbles under the skin that produce a crackling sound or crinkly feeling}
	}

	\newglossaryentry{cyanosis}
	{
		name=cyanosis,
		description={a blue-gray skin color that is caused by reduced level of oxygen in the blood}
	}

	\newglossaryentry{dcap_btls}
	{
		name=\acrshort{dcapbtls},
		description={a menemonic for assessment in which each area of the body is evaluated for \acrlong{dcapbtls}}
	}

	\newglossaryentry{diaphoretic}
	{
		name=diaphoretic,
		description={characterized by light or profuse sweating}
	}

	\newglossaryentry{diastolic_pressure}
	{
		name=diastolic pressure,
		description={the pressure that remains in the arteries during the relaxing phase of the heart's cycle (disatole) when the left ventricle is at rest}
	}

	\newglossaryentry{distracting_injury}
	{
		name=distracting injury,
		description={any injury that prevents the patient from noticing other injuries he or she may have, even severe injuries; for example, a painful femur or tibia fracture that prevents the patient from noticing back pain associated with a spinal fracture}
	}

	\newglossaryentry{focused_assessment}
	{
		name=focused assessment,
		description={a type of physical assessment typically performed on patients who have sustained nonsignificant mechanisms of injury or on responsive medical patients.  This type of examination is based on the chief complaint and focuses on one body system or part}
	}

	\newglossaryentry{frostbite}
	{
		name=frostbite,
		description={damage to tissues as the result of exposure to cold; frozen or partially frozen body parts are frostbitten}
	}
	
	\newglossaryentry{general_impression}
	{
		name=general impression,
		description={the overall initial impression that determines the priority for patient care; based on the patient's surroundings, the \acrfull{moi}, signs and symptoms, and the chief complaint}
	}

	\newglossaryentry{golden_hour}
	{
		name=Golden hour,
		description={the time from injury to defininitive care, during which treatment of shoc and traumatic injuries should occur because survivial potential is best; also called the Golden Period}
	}

	\newglossaryentry{guarding}
	{
		name=guarding,
		description={involuntary muscle contractions of the abdominal wall to minimize the pain of abdominal movement; a sign of peritonitis}
	}

	\newglossaryentry{history_taking}
	{
		name=history taking,
		description={a step within the patient assessment process that provides detail about the patient's chief complaint and an account of the patient's signs and symptoms}
	}

	\newglossaryentry{hypertension}
	{
		name=hypertension,
		description={blood pressure that is higher than the normal range}
	}

	\newglossaryentry{hypotension}
	{
		name=hypotension,
		description={blood pressure that is lower than the normal range}
	}

	\newglossaryentry{hypothermia}
	{
		name=hypothermia,
		description={condition in which the internal body temperature falls below 95°F (35°C)}
	}
	
	\newglossaryentry{incident_command_system}
	{
		name=\acrlong{ics},
		description={a system implemented to manage disasters and mass- and multiple-casualty incidents in which section chiefs, including finance, logistics, operations, and planning report to the incident commander}
	}
	
	\newglossaryentry{mechanism_of_injury}
	{
		name=\acrfull{moi},
		description={the forces, or energy transmission, applied to the body that cause injury}
	}
	
	\newglossaryentry{nature_of_illness}
	{
		name=\acrfull{noi},
		description={the general type of illness a patient is experiencing}
	}
	
	\newglossaryentry{opqrst_definition}
	{
		name=\acrfull{OPQRST},
		description={a mnemonic used in evaluating a pai}
	}	

	\newglossaryentry{orientation}
	{
		name=orientation,
		description={your evaluation of the conditions in which you will be operating}
	}
	
	% perfusion already defined in ch06
	
	\newglossaryentry{primary_assessment}
	{
		name=primary assessment,
		description={your evaluation of the conditions in which you will be operating}
	}
	
	\newglossaryentry{responsiveness}
	{
		name=responsiveness,
		description={The way in which a patient responds to external stimuli, including verbal stimuli (sound), tactile stimuli (touch) and painful stimuli}
	}
	
	\newglossaryentry{SAMPLE_history}
	{
		name=SAMPLE history,
		description={a brief history of a patient's condition to determine signs and symptoms, allergies, medications, pertinent past history, last oral intake, and events leading to the injury or illness}
	}
	
	\newglossaryentry{scene_sizeup}
	{
		name=scene size-up,
		description={A step within the patient assessment process that involves a quick assessment of the scene and the surroundings to provide information about scene safety and that \acrfull{moi} or \acrfull{noi}}
	}
	
	\newglossaryentry{secondary_assessment}
	{
		name=secondary assessment,
		description={A step within the patient assessment process in which a systematic physical examination of the patient is performed.  The examination may be a systematic exam or an assessment that focuses on a certain area or region of the body, often determined through the chief complaint}
	}
	
	\newglossaryentry{shallow_respirations}
	{
		name=spontaneous respiration,
		description={respirations characterized by little movement of the chest wall (reduced tidal volume) or poor chest excursion}
	}
	
	\newglossaryentry{sign}
	{
		name=sign,
		description={objective findings that can be seen, heard, felt, smelled, or measured}
	}

	\newglossaryentry{situational_awareness}
	{
		name=situational awareness,
		description={your evaluation of the conditions in which you will be operating}
	}

	\newglossaryentry{sniffing_position}
	{
		name=sniffing position,
		description={an upright position in which the patient's head and chin are thrust forward slightly forward to keep the airway open}
	}
	
	\newglossaryentry{spontaneous_respiration}
	{
		name=spontaneous respiration,
		description={breathing that occurs without assistance}
	}

	\newglossaryentry{standard_precautions}
	{
		name=standard precautions,
		description={protective measures that have traditionally been developed by the \acrfull{cdc} for use in dealing with objects, blood, body fluids, and other potential exposure risks of communicable disease}
	}
	
	\newglossaryentry{stridor}
	{
		name=stridor,
		description={high-pitched noise heard primarily on inspiration}
	}
	
	\newglossaryentry{tachypnea}
	{
		name=tachypnea,
		description={increased respiratory rate}
	}
	
	\newglossaryentry{triage}
	{
		name=triage,
		description={the process of establishing treatment and transportation priorities according to severity of injury and medical need}
	}

	\newglossaryentry{vasoconstriction} 
	{
		name=vasoconstriction,
		description={the narrowing of a blood vessel, such as with hypoperfusion or cold extremities}
	}	

	\newglossaryentry{vital_signs}
	{
		name=vital signs,
		description={the key signs used to evaluate the patients overall condition, including respirations, pulse, blood pressure, \acrfull{loc}, and skin characteristics}
	}

	
\chapter{Patient Assessment}

\subsection*{Abbreviations}
\begin{description}[leftmargin=!,labelwidth=\widthof{\bfseries ABCDEF}]
	\item [\acrshort{avpu}] 		\acrlong{avpu}
	\item [\acrshort{loc}] 			\acrlong{loc}
	\item [\acrshort{moi}] 			\acrlong{moi}
	\item [\acrshort{ppe}] 			\acrlong{ppe}
\end{description}\hfill \\

\subsection*{Definitions}
\begin{description}
	\item [\gls{accessory_muscles}] 				\glsdesc{accessory_muscles}
	\item [\gls{altered_mental_status}] 			\glsdesc{altered_mental_status}
	\item [\gls{auscultate}] 						\glsdesc{auscultate}
	
	\item [\gls{avpu_scale}] 						\glsdesc{avpu_scale}
	\item [\gls{auscultate}] 						\glsdesc{auscultate}
	\item [\gls{auscultate}] 						\glsdesc{auscultate}
	\item [\gls{auscultate}] 						\glsdesc{auscultate}
	\item [\gls{auscultate}] 						\glsdesc{auscultate}
	\item [\gls{auscultate}] 						\glsdesc{auscultate}
	\item [\gls{auscultate}] 						\glsdesc{auscultate}
	\item [\gls{auscultate}] 						\glsdesc{auscultate}
	\item [\gls{auscultate}] 						\glsdesc{auscultate}
	\item [\gls{auscultate}] 						\glsdesc{auscultate}
	\item [\gls{auscultate}] 						\glsdesc{auscultate}
	\item [\gls{auscultate}] 						\glsdesc{auscultate}
	\item [\gls{auscultate}] 						\glsdesc{auscultate}
	\item [\gls{auscultate}] 						\glsdesc{auscultate}
	\item [\gls{auscultate}] 						\glsdesc{auscultate}
	\item [\gls{auscultate}] 						\glsdesc{auscultate}
	\item [\gls{auscultate}] 						\glsdesc{auscultate}
	\item [\gls{auscultate}] 						\glsdesc{auscultate}
	\item [\gls{auscultate}] 						\glsdesc{auscultate}
	\item [\gls{auscultate}] 						\glsdesc{auscultate}
	\item [\gls{auscultate}] 						\glsdesc{auscultate}
	\item [\gls{auscultate}] 						\glsdesc{auscultate}
	\item [\gls{auscultate}] 						\glsdesc{auscultate}
	\item [\gls{auscultate}] 						\glsdesc{auscultate}
	\item [\gls{auscultate}] 						\glsdesc{auscultate}
	\item [\gls{focused_assessment}] 				\glsdesc{focused_assessment}
	\item [\gls{general_impression}] 				\glsdesc{primary_assessment}
	\item [\gls{hypothermia}] 						\glsdesc{hypothermia}
	\item [\gls{hypertension}] 						\glsdesc{hypertension}
	\item [\gls{mechanism_of_injury}] 				\glsdesc{mechanism_of_injury}
	\item [\gls{nature_of_illness}] 				\glsdesc{nature_of_illness}
	\item [\gls{orientation}] 						\glsdesc{orientation}
	\item [\gls{perfusion}] 						\glsdesc{perfusion}
	\item [\gls{primary_assessment}] 				\glsdesc{primary_assessment}
	\item [\gls{responsiveness}] 					\glsdesc{responsiveness}
	\item [\gls{SAMPLE_history}] 					\glsdesc{SAMPLE_history}
	\item [\gls{secondary_assessment}]				\glsdesc{primary_assessment}
	\item [\gls{scene_sizeup}] 						\glsdesc{scene_sizeup}
	\item [\gls{secondary_assessment}] 				\glsdesc{primary_assessment}
	\item [\gls{shallow_respirations}] 				\glsdesc{shallow_respirations}
	\item [\gls{sign}] 								\glsdesc{sign}
	\item [\gls{situational_awareness}] 			\glsdesc{situational_awareness}
	\item [\gls{sniffing_position}] 				\glsdesc{sniffing_position}
	\item [\gls{spontaneous_respiration}] 			\glsdesc{spontaneous_respiration}
	\item [\gls{standard_precautions}] 				\glsdesc{standard_precautions}
	\item [\gls{stridor}] 							\glsdesc{stridor}
	\item [\gls{tachypnea}] 						\glsdesc{tachypnea}
	\item [\gls{triage}] 							\glsdesc{triage}
	\item [\gls{vasoconstriction}] 					\glsdesc{vasoconstriction}
	\item [\gls{vital_signs}] 						\glsdesc{vital_signs}
\end{description}
	
\section{Introduction}
Patient assessment is divided into five (5) main parts:

\afterpage{%
\clearpage
\subsection{Potential Test Questions}
\begin{outline}[enumerate]
	\1 What is the difference between unconcious and unresponsive?
	\1[] 
	
	\1 What is the difference between unconcious and unresponsive?
	\1[] 
\end{outline}
}

\end{document}



\begin{document}
\setcounter{chapter}{12}
\label{ch:chapter13}
\clearpage
	
% Glossary acronym entries %
	\newacronym{abc}{ABC}{airway (obstruction) \newline breathing (respiratory arrest) \newline circulation (cardiac arrest)}
	% acronym {aed} defined in ch01
	\newacronym{afib}{a-fib}{atrial fibrillation}
	\newacronym{aha}{AHA}{American Heart Association}
	\newacronym{aicd}{AICD}{automated implanted cardioverter-defibrillator}
	% acronym {als} defined in ch01
	% acronym {bls} defined in ch01
	\newacronym{bvm}{BVM}{bag-valve mask}
	\newacronym{cpr}{CPR}{cardiopulmonary resuscitation}
	\newacronym{itd}{ITD}{impedance threshold device}
	% acronym {iv} defined in ch11
	\newacronym{ldb}{LDB}{load-distributing band}
	\newacronym{rosc}{ROSC}{return of spontaneous circulation}
	\newacronym{sca}{SCA}{sudden cardiac arrest}
	\newacronym{vfib}{v-fib}{ventricular fibrillation}
	\newacronym{vtac}{v-tac}{ventricular tachycardia}

% Glossary entries	
	\newglossaryentry{active_compression_decompression_CPR}
	{
		name=active compression decompression \acrshort{cpr},
		description={technique that involves compressing the chest and then actively pulling it back up to its neutral position and beyond}
	}

	\newglossaryentry{atrial_fibrillation}
	{
		name=\acrfull{afib},
		description={upper heart chambers contract irregularly}
	}

	% automated_external_defibrillator defined in ch01

	\newglossaryentry{anoxia}
	{
		name=anoxia,
		description={absence of oxygen}
	}

	\newglossaryentry{aortocaval_compression}
	{
		name=aortocaval compression,
		description={relating to the aorta and the vena cava}
	}

	\newglossaryentry{apneic}
	{
		name=apneic,
		description={absence of spontaneous breathing}
	}

	\newglossaryentry{automated_implanted_cardioverter_defibrillator}
	{
		name=\acrfull{aicd},
		description={the technical term for \emph{pacemaker}}
	}

	\newglossaryentry{basic_life_support}
	{
		name=\acrfull{bls},
		description={noninvasive, emergency lifesaving care that is used to treat medical conditions, including airway obstruction, respiratory arrest, and cardiac arrest}
	}

	\newglossaryentry{bradycardia}
	{
		name=bradycardia,
		description={slow heart rate}
	}

	\newglossaryentry{cardiopulmonary_resuscitation}
	{
		name=\acrfull{cpr},
		description={the combination of chest compressions and rescue breathing used to establish adequate ventilation and circulation in a patient who is not breathing and has no pulse}
	}

	% cyanosis defined in ch09

	\newglossaryentry{dyspnea}
	{
		name=dyspnea,
		description={difficulty or trouble breathing}
	}

	\newglossaryentry{fundus}
	{
		name=fundus,
		description={part of a hollow organ that is farthest from the opening}
	}

	\newglossaryentry{gastric_distention}
	{
		name=gastric distention,
		description={a condition in which air fills the stomach, often as a result of high volume and pressure during artificial ventilation}
	}

	\newglossaryentry{head_tilt_chin_lift_maneuver}
	{
		name=head tilt-chin lift maneuver,
		description={a combination of two movements to open the airway by tilting the forehead back and lifting the chin; not used for trauma patients}
	}

	\newglossaryentry{hypercarbia}
	{
		name=hypercarbia,
		description={increased level of carbon dioxide (\ensuremath{\mathrm{CO_2}}) in the bloodstream}
	}

	\newglossaryentry{hyperventilation}
	{
		name=hyperventilation,
		description={ Rapid or deep breathing that lowers the blood carbon dioxide (\ensuremath{\mathrm{CO_2}})  level below normal}
	}

	% hypotension defined in ch09

	\newglossaryentry{hypoxia}
	{
		name=hypoxia,
		description={a dangerous condition in which the body's tissues and cells do not have enough oxygen}
	}

	\newglossaryentry{impedance_threshold_device}
	{
		name=\acrfull{itd},
		description={a valve device placed between the endotracheal tube and a bag-valve mask that limits the amount of air entering the lungs during the recoil phase between chest compressions}
	}

	\newglossaryentry{intrathoracic}
	{
		name=intrathoracic,
		description={within the chest (thoracic) cavity}
	}

	\newglossaryentry{ischemia}
	{
		name=ischemia,
		description={decreased oxygen supply}
	}

	\newglossaryentry{jaw_thrust_maneuver}
	{
		name=jaw-thrust maneuver,
		description={technique to open the airway by placing the fingers behind the angle of the jaw and bringing the jaw forward; use for patients who may have a cervical spine injury}
	}

	\newglossaryentry{load_distributing_band}
	{
		name=\acrfull{ldb},
		description={circumferential chest compression device composed of a constricting band and backboard that is either electrically or pneumatically driven to compress the heart by putting inward pressure on the thorax}
	}

	\newglossaryentry{opiate}
	{
		name=opiate,
		description={A subset of the opiod family, referring to natural, non-synthetic opioids}
	}

	\newglossaryentry{opioid}
	{
		name=opiod,
		description={A synthetically-produced narcotic medication, drug, or agent similar to the opiate morphine, but not derived from opium.  Used to relieve pain}
	}

	\newglossaryentry{pacemaker}
	{
		name=pacemaker,
		description={a medical device that delivers shocks directly to the heart if necessary.\newline Also called an \acrfull{aicd}}
	}

	\newglossaryentry{stoma}
	{
		name=stoma,
		description={an opening through the skin and into an organ or other structure}
	}

	\newglossaryentry{tachycardia}
	{
		name=tachycardia,
		description={rapid heart rate ($>$ 100 beats/minute)}
	}

	\newglossaryentry{ventricular_defibrillation}
	{
		name=\acrfull{vfib},
		description={disorganized, ineffective quivering of the ventricles, resulting in no blood flow and a state of cardiac arrest}
	}

	\newglossaryentry{ventricular_tachycardia}
	{
		name=\acrfull{vtac},
		description={rapid heart rhythm in which the elctrical impulse begins in the ventricle (instead of the atrium), which may result in inadequate blood flow and eventually deteriorate into cardiac arrest}
	}

	\newglossaryentry{xiphoid_process}
	{
		name=xiphoid process,
		description={cartilaginous section at the lower end of the sternum}
	}

\chapter{\acrshort{bls} Resuscitation}

\subsection*{Abbreviations}
\begin{description}[leftmargin=!,labelwidth=\widthof{\bfseries ABCDE}]
	\item [\acrshort{abc}] 			\acrlong{abc}
	\item [\acrshort{aed}] 			\acrlong{aed}
	\item [\acrshort{aha}] 			\acrlong{aha}
	\item [\acrshort{aicd}] 		\acrlong{aicd}
	\item [\acrshort{als}] 			\acrlong{als}
	\item [\acrshort{bls}] 			\acrlong{bls}
	\item [\acrshort{bvm}] 			\acrlong{bvm}
	\item [\acrshort{cpr}] 			\acrlong{cpr}
	\item [\acrshort{itd}] 			\acrlong{itd}
	\item [\acrshort{iv}] 			\acrlong{iv}		% ch11
	\item [\acrshort{ldb}] 			\acrlong{ldb}
	\item [\acrshort{rosc}] 		\acrlong{rosc}
	\item [\acrshort{sca}] 			\acrlong{sca}
	\item [\acrshort{vfib}] 		\acrlong{vfib}
	\item [\acrshort{vtac}] 		\acrlong{vtac}
\end{description}

\subsection*{Definitions}
\begin{description}
	\item [\gls{active_compression_decompression_CPR}] 	\glsdesc{active_compression_decompression_CPR}
	\item [\gls{atrial_fibrillation}] 					\glsdesc{atrial_fibrillation}
	\item [\gls{automated_external_defibrillator}] 		\glsdesc{automated_external_defibrillator}
	\item [\gls{anoxia}] 								\glsdesc{anoxia}
	\item [\gls{aortocaval_compression}] 				\glsdesc{aortocaval_compression}
	\item [\gls{apneic}] 								\glsdesc{apneic}
	\item [\gls{automated_implanted_cardioverter_defibrillator}] 	\glsdesc{automated_implanted_cardioverter_defibrillator}
	\item [\gls{basic_life_support}] 					\glsdesc{basic_life_support}
	\item [\gls{cardiopulmonary_resuscitation}] 		\glsdesc{cardiopulmonary_resuscitation}
	\item [\gls{cyanosis}] 								\glsdesc{cyanosis}
	\item [\gls{diaphoretic}] 							\glsdesc{diaphoretic}
	\item [\gls{dyspnea}] 								\glsdesc{dyspnea}
	\item [\gls{fundus}] 								\glsdesc{fundus}
	\item [\gls{gastric_distention}] 					\glsdesc{gastric_distention}
	\item [\gls{head_tilt_chin_lift_maneuver}] 			\glsdesc{head_tilt_chin_lift_maneuver}
	\item [\gls{hypercarbia}] 							\glsdesc{hypercarbia}
	\item [\gls{hyperventilation}] 						\glsdesc{hyperventilation}
	\item [\gls{hypotension}] 							\glsdesc{hypotension}
	\item [\gls{hypoxia}] 								\glsdesc{hypoxia}
	\item [\gls{impedance_threshold_device}] 			\glsdesc{impedance_threshold_device}
	\item [\gls{intrathoracic}]							\glsdesc{intrathoracic}
	\item [\gls{ischemia}] 								\glsdesc{ischemia}
	\item [\gls{jaw_thrust_maneuver}] 					\glsdesc{jaw_thrust_maneuver}
	\item [\gls{load_distributing_band}] 				\glsdesc{load_distributing_band}
	\item [\gls{opiate}] 								\glsdesc{opiate}
	\item [\gls{opioid}] 								\glsdesc{opioid}
	\item [\gls{pacemaker}] 							\glsdesc{pacemaker}
	\item [\gls{stoma}] 								\glsdesc{stoma}
	\item [\gls{tachycardia}] 							\glsdesc{tachycardia}
	\item [\gls{ventricular_defibrillation}] 			\glsdesc{ventricular_defibrillation}
	\item [\gls{ventricular_tachycardia}] 				\glsdesc{ventricular_tachycardia}
	\item [\gls{xiphoid_process}] 						\glsdesc{xiphoid_process}
\end{description}\hfill \\
\clearpage

\subsection*{Heart terms}
\begin{description}
	\item [\gls{atrial_fibrillation}] \glsdesc{atrial_fibrillation}
	\item [\gls{bradycardia}] \glsdesc{bradycardia}
	\item [\gls{tachycardia}] \glsdesc{tachycardia}
	\item [\gls{ventricular_defibrillation}] \glsdesc{ventricular_defibrillation}
	\item [\gls{ventricular_tachycardia}] \glsdesc{ventricular_tachycardia}
\end{description}\hfill \\

\section{\acrfull{bls}}
\acrshort{bls}  is noninvasive, emergency lifesaving care that is used to treat medical conditions, including airway obstruction, respiratory arrest, and cardiac arrest.

\paragraph{\acrshort{bls} sequence} (use ABC mnemonic):
\begin{description}[leftmargin=!,labelwidth=\widthof{\bfseries respiratory}]
	\item [airway] (obstruction)
	\item [breathing] (respiratory arrest)
	\item [circulation] (cardiac arrest)
\end{description}

\paragraph{Difference between \acrshort{als} and \acrshort{bls}} \acrshort{als} involves advanced lifesaving procedures such as cardiac monitoring, administration of \acrshort{iv} fluids and medications, and the use of advanced airway adjuncts.

\paragraph{Permanent brain damage} is possible after only 4-6 minutes without oxygen. \newline
To survive cardiac arrest, effective \underline{\acrshort{cpr}} at an adequate rate and depth with minimal interruptions is essential \underline{until defibrillation can be administered}.\hfill \\


\section{\acrshort{bls} Procedures}
According to the \acrshort{aha} 88\% of sudden cardiac arrests occur in the home 

\subsection{The 'Chain of Survival'}
\begin{outline}{enumerate}
	\1 Recognition and activation of the emergency response system
		\2 Laypeople must recognize the early warning signs of cardiac emergency to call 9-1-1
		\2 Requires public education and awareness
	\1 Immediate high-quality \acrshort{cpr}
	\1 Rapid defibrillation 
		\2 \acrshort{aed} must be used as soon as it is available \textit{without stopping chest compressions}
	\1 basic and advanced emergency medical services
		\2 \acrshort{als}: high-quality \acrshort{cpr}, early defibrillation, and use of devices and/or drugs.
	\1 Advanced life support and post arrest care
		\2 comprehensive, multidisciplinary system of care including mild therapeutic hypothermia and other treatments
\end{outline}

\subsection{\acrshort{cpr} steps}
\begin{enumerate}
	\item Restore circulation by performing chest compressions to circulate blood.
	\item 100-120 chest compressions per minute for 2 minutes
	\begin{itemize}
		\item Depth of 2 inches to 2.4 inches (5 - 6cm)
		\item Open airway with the \underline{jaw-thrust} or \underline{head tilt-chin lift} maneuver
	\end{itemize}
	\item Restore breathing by providing rescue breaths via mouth-to-mask ventilation, or bag-valve mask (BVM) minister
	\begin{itemize}
		\item 2 breaths over 1 second while watching for chest rise.
	\end{itemize}
\end{enumerate}

\subsection{Differences in providing \acrshort{cpr} for infants, children and adults}
\begin{enumerate}
	 \item \acrshort{cpr} emergencies for infants and children require \acrshort{cpr} usually have different underlying causes
	 \item Anatomical differences: children and infants have smaller airways than adults
\end{enumerate}

\begin{description}[leftmargin=!,labelwidth=\widthof{\bfseries Children and infants}]
	\item [Adults:] usu. cardiac arrest \textrightarrow  respiratory arrest
	\item [Children \& infants:] usu. respiratory arrest \textrightarrow  respiratory arrest
\end{description}

\subparagraph{Complications from chest compressions} are rare but can include fractured ribs lacerated liver and a fractured sternum.\hfill \\


\section{Assessing the Need for BLS}
\subsection{When not to start \acrshort{cpr}}
\begin{enumerate}
	\item If the scene is unsafe
	\item If the patient has obvious signs of death (obv. mortal damage, dependent lividity, rigor mortis, putrefaction)
	\item If the patient/their physician has DNR or no \acrshort{cpr} order
\end{enumerate}

\subsection{Special \acrshort{aed} situations}
\subsubsection{automated implanted cardioverter-defibrillator (AICD)} 
Automated implanted cardioverter-defibrillators (AICD), commonly known as \textbf{pacemakers}, deliver shocks directly to the heart if necessary 
\paragraph{Identifying AICDs} AICDs create a hard lump beneath the skin on the \underline{upper-left side} of the chest (just below the clavicle)
\paragraph{\acrshort{aed} usage with \acrshort{aicd}s} 
\begin{itemize}
	\item[] \underline{Do not} pads directly over the device: this reduces effectiveness of \acrshort{aed} shock.
	\item Place \acrshort{aed} pads at least \underline{1 inch (2.5 cm) away} from the device.
	\item Occasionally, implanted device will deliver shocks to the patient 
	\begin{itemize} 
		\item If you observe the patient's \textbf{muscles twitching}: \newline
		\underline{continue \acrshort{cpr}} and wait \underline{30 -- 60 seconds before} delivering the shock from the \acrshort{aed}.
	\end{itemize}
\end{itemize}

\subsubsection{Patient's chest is wet}
If patient's chest is wet, the electrical current may move across the skin rather than between the pads.
\paragraph{Patient is submerged in water} 
\begin{enumerate}
	\item Pull patient out of the water
	\item Quickly dry skin \underline{before attaching} \acrshort{aed} pads 
	\item Do \textbf{not} delay \acrshort{cpr} to dry the patient thoroughly
	\begin{itemize}
		\item instead, quickly wipe off as much moisture as possible from the chest
	\end{itemize}
\end{enumerate}

\paragraph{Patient lying in small puddle of water or snow}
\begin{itemize}
	\item \acrshort{aed} can be used but again the patient's chest should be quickly dried as much as possible
\end{itemize}

\subsubsection{transdermal medication} patches you may encounter patient who is receiving medication through transdermal medication patch such as nitroglycerin if the medication patch interferes with \acrshort{aed} pad placement them remove the patch with your gloved hands and wiped the skin to remove any residue prior to attaching the \acrshort{aed} pad

\section{Devices and Techniques to Assist Circulation}
\subsection{active compression-decompression \acrshort{cpr}} 
Technique that involves compressing the chest and then actively pulling it back up to its neutral position and beyond.
May increase the amount of blood that returns to the heart and thus the amount of blood ejected from the heart during the compression phase
\subsection{impedance threshold device (IPD)}
Valve device placed between the ET tube and it BVM
may also be placed between the bag and mask if an ET tube is not in place
limits air entering the lungs during the \emph{recoil phase} between chest compressions
Results in \emph{negative} intrathoracic pressure that may draw more blood towards the heart ultimately resulting in improved cardiac filling and circulation.
it is not currently recommended for use with conventional \acrshort{cpr}
if our OSC occurs than the IPD should be removed
\subsection{mechanical piston device} device that depresses the sternum via plunger mounted on a backboard.
\paragraph{Positioning} supine on backboard \newline plunger centered over the patient's thorax (same place hands would go).

\subsection{LVAD}
The effectiveness of \acrshort{cpr} depends on the amount of blood circulated throughout the body as a result of chest compressions
before you consider the use of mechanical devices to assist circulation ensure that your manual chest compressions her concerns distantly high quality

\section{Special resuscitation circumstances} 
\subsection{Opioid overdose}
Narcotic that when taken in excess depresses the central nervous system causes respiratory arrest followed by cardiac arrest.
\paragraph{Naloxone} Chest compressions ventilation defibrillation take priority over naloxone 
administration do not delay other interventions while awaiting the patient's response to naloxone therapy 
May have a \textbf{pulse} but \textbf{not breathing}: BVM ventilation \emph{is the most critical treatment}, followed by naloxone (if available). \newline
\subsection{Pregnancy \& cardiac arrest}
\paragraph{Priority} provide high-quality \acrshort{cpr}.  Relieve pressure off the \emph{aorta} and \emph{vena cava}.
When patient lies supine, the pregnant uterus can \emph{compress the aorta and vena cava (aortocaval compression)} \newline 
If pregnant patient is \textbf{not} in cardiac arrest: position her on the \textbf{left side} to relieve pressure on the vessels aorta and vena cava. \newline

If she is in cardiac arrest, this is impractical.
because she must remain in the supine position to maximize effectiveness of compressions therefore if the top the patient's uterus (fundus) 
can be felt at or above the level of the umbilicus  perform manual displacement of the uterus to the patient's left to relieve aorta painful compression while \acrshort{cpr} is being performed 

Whenever you assist the patient remember that his or in some patient apps that in some cases family members may experience a psychological crisis that turns to medical crisis and may become patients themselves 

\begin{table}[ht]
	\centering
	\caption{Review of BLS procedures}
	
	\bgroup
	\def\arraystretch{1.25}%
	\begin{tabular}{|p{3cm}|p{4cm}|p{4cm}|p{4cm}|}
		\hline
		\bfseries Procedure						 & \bfseries Adult                                        & \textbf{Child} \newline \small {Age 1 month -- 1 year} 					   			
																																						    &  \textbf{Infant} \newline \small {Age 1 year -- onset of \newline puberty} \\ \hline \hline
		\multicolumn{4}{l}{\bfseries Circulation} \\ \hline
		pulse check                              & carotid artery                                         & carotid or femoral artery 		   				& brachial artery \\ \hline
		compression area                         & center of the chest, in between the nipples     		  & center of the chest, in between the nipples 	& just below the nipple line \\ \hline
		compression width						 & heel of both hands									  & heel of one or both hands 						& Two-finger technique, \newline or two-thumb encircling \newline-hands technique \\ \hline
		compression depth                        & 2 -- 2.4 in. \newline (5 -- 6 cm)                      & at least \( \frac{1}{3} \) anterior-posterior diameter \newline (\textasciitilde 2 in. or 5cm) 
																																							& At least \( \frac{1}{3} \) anterior-posterior diameter \newline (\textasciitilde 1.5 in. or 4 cm) \\ \hline
		compression rate                         & \multicolumn{3}{c|}{100 to 120/min} \\ \hline
		compression-to-ventilation ratio*        & \multicolumn{3}{c|}{10:1}  \\ \hline
		Foreign body \newline obstruction        & \textbf{Responsive:} \newline abdominal thrusts (Heimlich); \newline chest thrust if the \newline patient is pregnant or has obesity \newline  \textbf{Unresponsive}: \acrshort{cpr} 
																	 									  & \textbf{Responsive:} \newline abdominal thrusts (Heimlich) \newline \newline \newline \newline \textbf{Unresponsive:} \acrshort{cpr}
																	 									   										 			& \textbf{Responsive:} \newline back slaps, chest thrusts \newline \newline \newline \newline \newline \textbf{Unresponsive:} \acrshort{cpr} \\ \hline
		\multicolumn{4}{l|}{\bfseries Airway} \\ \hline
		airway \newline positioning 			 & \multicolumn{3}{c|}{head tilt-chin lift\texttt{;} \newline jaw-thrust if spinal injury suspected} \\ \hline
		\multicolumn{4}{l}{\bfseries Breathing} \\ \hline
		ventilations                             & 1 breath every 5 -- 6 sec. (10 -- 12 breaths/min); \newline \textasciitilde 1 second per breath; \newline visible chest rise
																										  & 1 breath every 3--5 seconds \newline (12 to 20) breaths/min; \newline \textasciitilde 1 second per breath; \newline visible chest rise
																																			 	 & 1 breath every 3--5 seconds \newline (12 to 20) breaths/min; \newline \textasciitilde 1 second per breath; \newline visible chest rise \\ \hline
		ventilations (with advanced airway placed) & 1 breath every 6 seconds (rate of 10 breaths/min)      & 1 breath every 6 seconds (rate of 10 breaths/min) 
																																				 & 1 breath every 6 seconds (rate of 10 breaths/min) \\ \hline
	\end{tabular}
	\egroup
\end{table}\hfill \\
\clearpage

\subsection{Potential Test Questions}
\begin{enumerate}
	\item How does BLS differ from \acrshort{als}? \hfill \\
	\acrshort{als} involves advanced lifesaving procedures such as cardiac monitoring, administration of  IV fluids and medications, and the use of advanced airway adjuncts.
	\item What is the difference between hypoxia and ischemia? \hfill \\
	Hypoxia is when oxygen saturation is below 90\%, while ischemia is when blood supply to tissue is interrupted. \newline
	ischemia leads to hypoxia.
	\item What conditions must be present for a patient to be placed into the recovery position? 
	Unconscious, no traumatic injuries, breathing on their own.\hfill \\
	\item Under what circumstances does an \acrshort{emt} \textbf{not} start \acrshort{cpr}?
	\begin{itemize}
		\item If the scene is unsafe
		\item If the patient has obvious signs of death (obv. mortal damage, dependent lividity, rigor mortis, putrefaction)
		\item If the patient/their physician has DNR or no \acrshort{cpr} order
	\end{itemize}\hfill \\
\end{enumerate}


\begin{comment}
\paragraph{Special circumstances}
\begin{enumerate}
\item Determine unresponsiveness:
\begin{description}[leftmargin=!,labelwidth=\widthof{\bfseries respiratory}]
\item [adults] Gently tap the patient on the shoulder, shout, "Are you okay?"
\item [infants] (respiratory arrest)
\item [children] (cardiac arrest)
\end{description}
\end{enumerate} \hfill\\

One recognition and activation of the emergency response system the first step in the chain of survival requires
immediate high-quality CPR the initiation of immediate CPR by bystanders essential for successful resuscitation of a person cardiac arrest
three rapid defibrillation provided that immediate high-quality CPR with minimum interruptions is performed early defibrillation offers the best opportunity to achieve a successful patient outcome
four basic and advanced emergency medical services this link in the chain describes care by EMTs and ALS providers before the patient arrives at the ED such care includes continuing high-quality CPR basic airway management such as oral airway insertion and BVM ventilation advanced airway management endotracheal intubation or use of supper logic airway devices many will defibrillation vascular access transcutaneous pacing and administration of medications
five advanced life support and postarrest care after team delivers the patient to the emergency room further cardiopulmonary and neurologic support is provided to improve the patient's recovery when indicated this support can include additional medication therapy to support blood pressure

the first step is to determine unresponsiveness gently tap the patient on the shoulder and shouted are you okay
continue your assessment by simultaneously checking for breathing in the pulse the step should take no more than 10 seconds
if the patients in cardiac arrest they begin CPR immediately
the basic principles of BLS are the same for infants children and adults
for the purposes of BLS anyone younger than one year is considered an infant
a child is between one year of age and the onset of puberty approximately 12 to 14 years of age as signified by breast development in girls and underarm chest and facial hair and boys
some small children made best be treated as infants some larger children as adults

Most prehospital cardiac arrests occur as the result of a sudden cardiac rhythm disturbance is arrhythmia such as ventricular fibrillation V. fib or pulseless in curricular tachycardia VTEC
the normal heart rhythm is known
as normal sinus rhythm V. fib is the disorganized quivering of the ventricles
resulting in no forward blood flow in the state of cardiac arrest VTEC is a rapid contraction of the ventricles that this not allow for normal filling of the heart AEDs can beasts safely used in children using the pediatric sized pads and a dose attenuating system energy reducer
however if these items are unavailable use adult sized AED pads apply the AED to infants or children after the first five cycles of CPR have been completed recall that cardiac arrest in children is usually the result of respiratory failure therefore oxygenation ventilation are vitally important after the first five cycles CPR using AED to deliver shocks in the same manner as with them adult patient

If the patient is an infant between one month and one year of age the manual defibrillators preferred to an AED however this is an ALS school if you use adult sized AED pads on an infant or small child and do not cut the pads to adjust the size

Positioning the patient for CPR to be effective the patient must be lying supine on a firm flat surface with enough clear space around the patient for two rescuers to perform CPR and use the AED
the patient is crumpled up her lying face down then you will need to move him or her to a supine position
be mindful that you cannot rule out a spinal injury and an unresponsive patient therefore protect the patient's neck and move her or him as a unit without twisting the patient's on the bed the move him or her to the floor if possible log roll the patient onto a long backboard as you position him or her for CPR do this as quickly and safely as possible a backboard will provide support during transport and emergency care

Checking for breathing and a pulse after you have determined that the patient is unresponsive quickly check for breathing and a pulse should take no longer than 10 seconds in total
visualize this chest for signs of breathing while palpating for a cardioid pulse feel for the cardio artery by locating the larynx at the front of the neck and sliding two fingers towards one side the side closest to you

If the patient is not breathing or is breathing only slowly or occasionally known as agate
and does not have a pulse to begin CPR starting with chest compressions depression squeeze the heart thereby act acting as of pump to circulate blood

1.	 take standard precautions 
2.	waste the heel of one hand on the center of the chest over the lower half of the sternum
3.	place the heel of your other hand over the first hand for with your arm straight lock your elbows and position your shoulders directly over your hand so that the thrust of each compression is straight down on the sternum
5. Depress the sternum to a depth of 2 inches to 2.4 inches 5 cm to 6 cm using direct downward movement and then rising gently upwards
the ratio of time devoted to compression versus relaxation should be 1 to 1


Your motions must be smooth rhythmic and uninterrupted short jabbing compressions are not effective in producing artificial blood flow

Chest compressions create blood flow to the heart through filling of the coronary arteries
it takes 5 to 10 compressions to reestablish effective blood flow to the heart after chest compressions her resume
opening the airway in adults the two techniques for opening the airway in adults with a head tilt chin lift maneuver in the jaw thrust maneuver
these manual maneuvers are designed to bring the tongue forward and off the back of the throat

The head tilt chin lift maneuver is effective for opening the airway and most patients with there is no indication of a spinal injury

The patient is any foreign material or vomitus in the mouth and quickly remove it remove any liquid materials from the mouth with the suction device use your hooked finger to remove any solid material

If spinal injury suspected that use the jaw thrust maneuver do not tilt the patient's head back because you want to minimize the movement of the patient's neck to perform a jaw thrust maneuver place your fingers behind the angles of the patient's lower jaw and then move the jaw upward
keep the head in a neutral position as you move the jaw upward to close the mouth

If the jaw thrust fails to open airway than that head tilt chin lift should be used to open the airway

Recovery position if the patient is breathing adequately on his or her own and is no signs of injury to the spine or pelvis or hip and place him or her in the recovery position this position helps to maintain a clear airway in the patient with a decreased level of consciousness who is not sustained traumatic injuries and is breathing adequately on his or her own it also allows vomitus to drain from the mouth
roll the patient onto his or her sides that the head shoulders and torso move as a unit without twisting them placed the top hand under his or her cheek never place a patient who has a suspected head injury spinal injury in their current position because of this position the spine is not aligned spinal stabilization is not possible and further spinal injury could result

A lack of oxygen hypoxia combined with too much carbon dioxide in the blood hypercardia via is lethal to correct this condition you must provide slow deliberate ventilations that last one second this gentle slow method of ventilating the patient prevents the air from being forced into the stomach

Ventilations can be given by one or two EMS providers use a barrier device know yet he devices with an oxygen reservoir will provide higher percentages of oxygen to the patient regardless of whether you ventilate the patient with or without supplemental oxygen you should observe the chest for pop
visible rise to assess the effectiveness of your ventilations
hyperventilation may cause increased intra-thoracic pressure pressure inside the chest cavity
increased intrathoracic pressure reduces the amount of blood that returns to the heart thus decreasing the effectiveness of chest compressions are resulting in the heart and brain receiving decreased amounts of oxygen

Stoma ventilation
patients who have undergone old layering ectomy surgical removal of the larynx often have a permanent tracheal stoma in the midline of the neck in this case system is an opening that connects the trachea directly to the skin

Because it is at the midline the stoma is the only opening that will move air into the patient's lungs patients with a stoma should be ventilated with the BVM or pocket mask device placed directly over the stoma


Not all stoma's are disconnected from the nose and mouth if air leakage through the nose and mouth the interferes with ventilation through the stoma than cover the nose and mouth with your hand to make a seal
use a pediatric or infant master ventilate through the stoma

Ventilation is the physical act of moving air in and out of the lungs
ventilation is required for adequate respiration
examples of conditions that hinder ventilation include trauma such as flail chest foreign body airway obstruction and an injury to the spinal cord that disrupts the phrenic nerve that innervates the diaphragm.

Gastric distention
artificial ventilation may result in the stomach becoming filled with air
a condition called gastric distention
although it occurs more easily in children this condition also happens frequently in adults gastric distention is likely to occur if you hyperventilate the patient
you ventilate to forcefully of the patient's airway is not opened adequately in the excess gas opens up in the collapsible to the esophagus and allows gas to enter the stomach if massive gastric distention interferes with adequate ventilation and contact medical control
check the airway again and reposition the patient watch for the rise and fall of the chest and avoid getting forceful breaths having a suction unit available case the patient vomits remember mortality increases significantly if aspiration occurs
if an ALS providers available than he or she can insert on borrow gastric or nasogastric tube to decompress the stomach

Two rescuer adult CPR two rescuer CPR is always preferable because it is less tiring and it facilitates effective chest compressions in fact the team approached the CPR and AED uses far superior to the one rescuer approach
take standard precautions establish unresponsiveness while your partner moves to the patient side to be ready to deliver chest compressions
two if the patient is unresponsive that simultaneously check for breathing and palpate for
if the patient is unresponsive that simultaneously check for breathing and palpate for a carotid pulse take no more than 10 seconds to do this three if the patient is not breathing and has no pulse to begin CPR study with chest compressions give 30 chest compressions at a rate of 100 220 per minute
Vinay D is available than apply it and follow its voice prompts do not interrupt chest compressions to apply the a
pads
for open airway according to your suspicion of spinal injury five give two ventilations of one second each
observer visible chest rise six perform five cycles of 30 compressions in two ventilations
this should take about two minutes
after two minutes of CPR the compressor ventilator should switch positions the switch time should take no longer than five seconds reanalyze the patient's
cardiac rhythm with the AED every two minutes and deliver a shock if indicated continue cycles of 30 chest compressions a two ventilations until ALS providers take over the patient starts to move

When CPR is in progress on the patient who has an advanced airway device in place i.e. ET tube King LT super aquatic airway I gel super aquatic airway cycles of CPR not indicated compression should be at the continuous rate of 100 220 per minute and ventilation should occur at a rate of one breath every six seconds do not attempt to synchronize compressions and ventilations do not pause between compressions to deliver breaths

Switching positions is critical to switch rescuers during CPR to maintain high quality compressions

DL VAD is a mechanical pump that is implanted in the chest and helps pump blood from the left ventricle to the aorta a tube from the device passes through the skin and is attached to an external power source that the patient wears on their belt
over the shoulder harness the L VAD is commonly implanted in patients with severe heart failure are those who are awaiting a heart transplant
the if the L VAD is working that you will hear a humming sound when listening to the chest with a stethoscope
blood throws that blood flows continuously through the L VAD and more assistance the L VAD is providing to the heart the weaker the patient's pulse will be
you should know the location of L VAD patients in your service area
L VAD coordinators are usually available for consult 24 hours a day these medical professionals typically work at the same facility that placed the device so they should also be familiar with the patient failure local protocols or contact medical control regarding the treatment of a patient with an L VAD


Infant and child CPR cardiac arrest in infants and children follows respiratory arrest which triggers hypoxia and ischemia decreased oxygen supply of the heart
children consume oxygen to the three times as rapidly as adults

So you must first focus on opening the airway and providing artificial ventilation
often this will be enough to allow the child to resume spontaneous breathing and thus prevent cardiac arrest airway and breathing are the focus of pediatric BLS

As discussed in chapter 34 pediatric emergencies respiratory issues leading to cardiopulmonary arrest in children have a number of different causes including
injury both blunt and penetrating infections of the respiratory tract and or another organ system group epiglottis epiglottis

A foreign body in the airway
submersion drowning electrocution

Poisoning or drug overdose sudden infant death syndrome
SIDS
determining responsiveness never shake a child to determine whether or not he or she is responsive especially at the possibility of a neck or back injury exists instead simply gently tap the child on the shoulder and say loudly are you okay
with an infant gently tap the soles of their feet
if a child is responsive but struggling to breathe and allow him or her to remain in whatever position is most comfortable
if you find an unresponsive apnea and pulseless child why you are alone and off duty and you did not witness the child's collapse form CPR begin with chest compressions for approximately five cycles about two minutes and do not and then stopped to call 911 and retrieve and AED do not call 911 right away as you would with an adult remember that cardiopulmonary arrest in children to smoke most often the result of respiratory failure and not a primary cardiac event
therefore children will require immediate restoration of oxygen ventilation and circulation which can be accomplished by immediately performing five cycles about two minutes of CPR before activating the EMS system
may encounter child whose cardiac arrest was caused by a primary cardiac event rather than the respiratory problem
although it is rare and otherwise healthy child without apparent respiratory condition suddenly collapses and you witness it then first confirm that the child is in cardiac arrest
if you are alone without a mobile phone and leave the
child to call 911 and get an AED before beginning CPR if you are not alone and send someone to call 911 and getting a DUI you begin CPR
the sun collapse of an otherwise healthy child does not indicate a respiratory problem instead it suggest a primary cardiac event that may responsive defibrillation therefore it is critical to get the A/D to the child side as soon as possible

After you establish responsiveness you need to assess breathing and circulation as with an adult this assessment can occur simultaneously and should take no longer than 10 seconds visualize the chest for signs of breathing and palpate for a pulse in a large central artery

You can usually palpate the carotid or for moral artery in children older than one year but it is difficult in infants and infants palpate the brachial artery which is located on the inner side of the arm
midway between the elbow and shoulder
place your thumb on the outer surface of the arm between the elbow and shoulder then placed the tips of your index and middle fingers on the inside of the biceps and press lightly on the bone toward the bone
CPR will be required if the infant or child is not breathing there is not breathing normally act in all gasps and the pulses absent or less than 60 beats a minute
as with an adult and infant or child must be lying on a hard flat surface for effective chest compressions

If you need to carry and fit while providing CPR than your forearm and hand can service the flat surface use your Palm to support the infant's head
the liver
is relatively large and fragile especially in infants
the spleen on the left is smaller and more fragile in children than in adults these organs are easily injured if you are not careful in performing chest compressions to be sure your hand position is correct before you begin
the chest of an infant is smaller more pliable than that of an older child or adult therefore you should only use two fingers to compress the chest
if two rescuers are performing CPR on an infant used it to thumb encircling hands technique to deliver the chest compressions in children especially those older than eight years you can use the heel of one or both hands to compress the chest

follow these steps to perform infant chest compressions:
1 Take Standard Precautions Pl. the infant on a firm surface using one hand to keep the head in an open airway position you can also use a patter wedge under the shoulders and upper body to keep the head from tilting forward
two
imagine a line drawn between the nipples placed two fingers in the middle of the sternum just below the nipple line 3 using two fingers compress the sternum at least one third of the anterior posterior diameter of the chest approximately 1.5
inches or 4 cm in most infants compress the chest at a rate of 100 to 120 per minute after each compression allow the sternum to return briefly to its normal position allow equal time for compression and relaxation of the chest not remove your fingers from the sternum and avoid jerky movements

Coordinate compressions and ventilations in the 30 to 2 ratio if you're working alone and 15 to 2 if you're working with a trained bystander another healthcare provider ensure the infant's chest fully recoils in between compressions and that the chest visibly rises with each compression you will find this easier to do if you use your free hand to keep the head in the open airway position
if the chest does not rise or rises only little then use the head tilt chin lift to open the airway reassess the infant for signs of spontaneous breathing or pulse after five cycles about two minutes of CPR



Shows the steps for performing CPR in children between one year of age and the onset of puberty 
one take standard precautions to place the child on a firm surface
place the heel of one or two hands in the center of the chest in between the nipples avoid compression over the lower tip of the sternum which is called the xiphoid process.
Two compress the chest at least one third of the anterior posterior diameter of the chest parenthesis approximately 2 inches 5 cm in most children parenthesis at a rate of 100 to 120 per minute with pauses for ventilation the actual number of compressions delivered will be about 80 per minute in between compressions allow the chest to fully recoil do not lean on the chest compression relaxation time should be the same duration using smooth movements use smooth movements hold your fingers off the child's ribs and keep the heel of your hands on the sternum for
four coordinate compressions and ventilations in the 30 to 2 ratio for one rescuer and 15 to 2 for two rescuers making sure the chest rises with each ventilation at the end of each cycle positive for two ventila 5:05 cycles about two minutes reassess for a pulse if there is no pulse and you have an AED continue CPR and apply the AED pads 
six if the child regains a pulse of greater than 60 bpm and resumes effective breathing place him or her in a position that allows for frequent reassessment of the airway and vital signs during transport 

Switching rescuer positions is the same for children as it is for adults every five cycles to minutes of CPR 
member if the child is passed onset of puberty use an adult CPR sequence including the use of an AED 
airway infants and toddlers often put toys and other objects as well as food in their mouth therefore foreign body obstruction of the upper airways common 
you must make sure the upper airways open when managing pediatric respiratory emergencies or cardiopulmonary arrest 
if the child is unresponsive in line in the supine position than the airway may become obstructed when the tongue and throat muscles relax in the tongue falls backward 


If child is unresponsive but breathing adequately to place him or her in the recovery position to maintain open airway and a large range of saliva vomitus or other secretions from the mouth 
do not use this position if you suspect an injury to the spine hips and pelvis unless you can secure the child to a backboard that can be tilted to the side the childish response in breathing but in a labored fashion to provide prompt transport to the closest appropriate hospital 

Opening the airway in an infant or child is done by using the same techniques as used for an adult however because the child's neck is so flexible the technique should be lightly however because a child's neck is so flexible the technique should be slightly modified the jaw thrust maneuver is the best method to use if you suspect a spinal injury in a child the second rescuers present here she should immobilized a child's cervical spine for spinal injury is not suspected then use the head tilt chin lift maneuver but modified so that just as you tilt the head back you are moving it only in the neutral position or a slightly extended position do not overextend the neck 

Head tilt chin lift maneuver
perform the head tilt chin lift maneuver in a child in the following manner
one
place one hand on the child's forehead and tilt the head back gently with the neck slightly extended to place two or three fingers not the thumb of your other hand under the child's chin and lift the jaw upward and outward do not close the mouth or push under the chin either move
may obstruct rather than open the airway three remove any visible foreign body or vomitus


Jaw thrust maneuver perform the jaw thrust maneuver in a child in the following manner place two or three fingers under each side of the angle of the lower jaw lift the jaw upward and outward if the jaw thrust alone does not open the airway in the cervical spine injury is not a consideration than tilt the head slightly if cervical spine
injury is suspected then use a second rescuer to immobilize the cervical spine
member that the head of the infant or young child is disproportionately large in comparison with the chest and shoulders as a result when a child is lying flat on his or her back especially on the backboard the head will bend forward hyper flexion
onto the upper chest this position can partially or completely obstruct the upper airway to avoid this possibility place a wedge of padding under the child's upper chest and shoulders

Provide rescue breathing if the child is not breathing but has pulse than open the airway and deliver one breath every 3 to 5 seconds
12 to 20 breaths a minute
the child is not breathing and does not have a pulse and deliver to rescue breaths after every 30s chest compressions 15 chest compressions of two rescuers at present
each ventilation should last about one second to produce a visible chest rise use the proper size mask and ensure an adequate mask to face seal


As infant or small child is breathing then provide prompt transport began a child who is in respiratory distress should be allowed to stay in whatever position is most comfortable children who are unresponsive but breathing with difficulty should be kept in a position it allows you to manage the airflow and provide ventilatory if needed.
In a child with such tracheostomy tube in the neck remove the mask from the BVM and connect it directly to the tracheostomy tube to ventilate the child the BVM is unavailable a facemask with one-way valve or other barrier device over the tracheostomy site can be used place your hand firmly over the child's mouth and nose to prevent the artificial breaths from leaking out of the upper airway

An injured child with a serious airway or breathing problem is likely to need full-time attention from two EMTs therefore it is important for you to arrange for backup from another unit as soon as possible perhaps even before you arrive on scene
in such cases you will need a driver and often additional help with patient care

Interrupting CPR
CPR is a crucial life-saving procedure that provides minimal circulation and ventilation until the patient can receive defibrillation
ALS treatment and definitive care at the ED no matter how well CPR is performed however it is rarely enough to save a patient's life if ALS is unavailable at the scene you must provide transport based on your local protocols and continue CPR on the way en route to the ED consider requesting a rendezvous with ALS providers if available this will provide ALS care to the patient sooner improving his or her chance for survival however not all EMS systems have ALS support available to them especially in rural settings
try not to interrupt CPR for more than a few seconds except when absolutely necessary for example if you too impatient up and down stairs
chest compression fraction is the total percentage of time during the resuscitation attempt in which chest compressions are being performed make every effort to maintain chest compression fraction of at least 60%
the higher the better the more frequent interruptions and chest compressions the lower the compression fraction will be low compression fractions lead to worse patient outcomes

You may also encounter physician orders for life-sustaining treatment PO LST or medical orders for life-sustaining treatment MOLST
forms these legal documents describe acceptable interventions for the patient in the form of medical orders and must be signed by an authorized medical provider to be valid
in all other cases begin CPR on anyone who is in cardiac arrest

When to stop CPR as an EMT you are generally not responsible for making the decision to stop CPR after you begin CPR in the field you must continue until one of the following events occurs the STO P mnenonic
S the patient starts breathing and has a pulse
T
the patient's care is transferred to another provider of equal or higher level training oh you are out of strength or too tired to continue CPR
P a physician who is present or providing online medical direction assumes responsibility for the patient and directs you to discontinue CPR


Out of strength does not merely mean weary rather it means that you are no longer physically able to perform CPR
in short always continue CPR until the patient's care is transferred to a physician or higher medical authority in the field in some cases your medical director or a designated medical
control physician may order you to stop CPR on the basis of patient's condition
patients who do not achieve patients who do not achieve our OSC may be put to natural kidney or liver donors in select situations
follow your local protocols regarding the care of potential organ donors
if you choose not to start CPR on a patient cardiac arrest and always comply with your local protocols and provide detailed documentation
in particular record the physical examination signs that let you decision and reference the protocol that states the signs are reason to not start CPR if special circumstances physically prevent you from making resuscitation attempts then document the scene conditions thoroughly

Foreign body airway obstruction in adults patiently a large foreign body will be aspirated and blocked upper airway and airway obstruction that an airway obstruction may be caused by various factors including relaxation of throat muscles and unresponsive patient vomited to regurgitate stomach contents blood damaged tissue after an injury dentures or foreign body such as food or small objects large objects that are visible but cannot be removed from the airway with suction such as loose dentures large pieces of food or blood clot should be's leapt forward and out with your gloved index finger so she could then be used as needed to keep the airway clear of thinner secretion such as blood from this mucus

An airway obstruction by foreign body in an adult usually occurs during a meal
in children it usually occurs during mealtime are at play

If the foreign body is not removed quickly then the lungs were use of their oxygen supply and unconsciousness and death will follow manager is based on the severity of the airway obstruction mild airway obstruction patients with a mild partial airway obstruction are still able to exchange adequate amounts of air but still have signs of respiratory distress

Breathing may be noisy however patient usually has strong effective cough
leave these patients alone your main concern is to prevent a mild airway obstruction from becoming a severe complete airway obstruction abdominal thrusts are not indicated for patients with a mild airway obstruction

Very patient with a mild airway obstruction first encourage him or her to cough or to continue coughing if they are already doing so do not interfere with the patient's own attempts to expel the foreign body instead give supplement auction if needed and provide prompt transport to the emergency department

Closely monitor the patient observe for signs of severe airway obstruction week or absent cough
increasing levels of consciousness cyanosis
responsive patient's son severe airway obstruction is usually easy to recognize
in someone who is eating or is just finished eating person is suddenly unable to speak or cough graphs his or her throats

Turn cyanotic and makes exaggerated efforts to breathe either air is not moving in to and out of the airway or the air movement is so slight it is not detectable at first the patient will be responsive and be able to clearly indicate the problem as the patient are you choking the patient with you soon
there's a minimal amount of air movement then you may hear a high-pitched sound called stridor
this occurs when the object is not fully blocking the airway but small amount of air entering the lungs is not enough to sustain life in the patient will eventually become unconscious if the obstruction is not relieved


Unresponsive patients
when you discover and unresponsive patient your first step is to determine whether he or she or brute is breathing or has a pulse you should suspect in airway obstruction if the standard maneuvers to open the airway and ventilate the lungs are ineffective

If you feel resistance when attempting to ventilate then the patient probably have some type of obstruction
removing a foreign body airway obstruction in an adult the manual maneuver recommended for removing severe airway risk obstructions and responsive adults and children older than one year is the abdominal thrust maneuver also called the Heimlich maneuver
the Heimlich technique creates an artificial cough by causing a sudden increase in the intrathoracic pressure when thrusts are applied to this subdiaphragmatic region is a very effective method for removing a foreign body obstruction from the airway

Responsive patients

Abdominal thrust maneuver the goal of the abdominal thrust maneuvers to compress the lungs upward and force residual air from the lungs to flow upward and expelled the object
in response patients with a severe airway obstruction
repeat abdominal thrust until the foreign bodies expelled of the patient becomes unresponsive each thrush to be deliberate with the intent of relieving the obstruction to reform abdominal thrust in a responsive adult use the following technique

One stand behind the patient and wrap your arms around his or her abdomen straddle your legs
outside the patient's legs this will allow you to easily slide the patient to the ground if he or she becomes unresponsive to make a fist with one hand grabbed the fist with the other hand placed the thumb side of the fist against the patient's abdomen just above the umbilicus and well below the xiphoid process

Three press your fist into the patient's abdomen with the quick inward and upward thrust for continue the abdominal thrust until the object is expelled from the airway or the patient becomes
unresponsive
chest thrusts you can perform the abdominal thrust maneuver safely on all adults and children however for women in advanced stages of pregnancy and for patients who have
obesity use chest thrusts instead to perform chest thrusts on the responsive adult use the following technique


One stand behind the patient with your arms directly under the patient's armpits and wrap your arms around the patient's chest to
make a fist with one hand grasp the fist with the other hand placed the thumb side of your fist against the patient's sternum avoiding the xiphoid process and the edges of the rib cage
three press your fist into the patient's chest with backward thrusts until the object is expelled or the patient becomes unresponsive or if the patient becomes unresponsive then begin CPR study with chest compressions

Responsive patients who become unresponsive

Patient with narrative structure may become unresponsive
in this case begin CPR starting with chest compressions use the following steps to manage the patient's airway obstruction wine carefully support the patient to the ground and immediately call for help
or send someone to call for help
to perform 30 chest compressions using the same landmark as you would for CPR center of the chest between the nipples do not check for a pulse before performing chest compressions
three open the airway and look in the mouth if he could see an object that can easily be removed and remove it with your fingers and attempt to ventilate if you do not see an object and resume chest compressions
for repeat steps two and three until the obstruction is relieved of the ALS providers take over


If you are able to remove an object from the mouth and attempt to ventilate if ventilation produces chest rise to continue to ventilate and check for a pulse

If a pulse is present but the patient is not breathing the continue rescue breathing and monitor the pulse if a pulse is absent they continue CPR compressions a ventilation
and apply ADD as soon as it is available
unresponsive patients when a patient is on unresponsive
is unlikely you will know what is cause the problem
begin CPR by determining unresponsiveness and checking for breathing and a pulse
if a boss is present but breathing is absent then open the airway and attempt to ventilate if the first ventilation does not produce visible chest rise then put position reposition the airway and attempt to ventilate if both ventilation attempts do not produce visible chest rise and perform 30 chest compressions and then open the airway and look in the mouth

If an object is visible and can easily be removed and remove it with your fingers and attempt to ventilate
never perform blind finger sweeps on any patient doing so may push the obstruction further into the airway for an object is not visible or cannot easily be removed and resume chest compressions
continue sequence of chest compressions opening the airway and looking inside the mouth until they were is clear ALS providers arrive

Foreign body airway obstruction in infants and children as mentioned previously airway obstruction is a common problem in infants and children usually caused by foreign body such as food or a toy or by an infection
resulting in swelling and are narrowing of the airway
try to identify the cause of the instruction as soon as possible patients who have signs and symptoms of airway infection do not waste time trying to dislodge a foreign body administer supplemental oxygen if needed and immediately transport the child to the emergency department

A previously healthy child who's eating or playing with small toys are an infinite was crying about the house and recently has difficulty breathing is probably aspirated a foreign body
as an adult's foreign bodies may cause a mild or severe airway obstruction with a mild airway obstruction the child can cough forcefully although he or she may wheeze between coughs as long as the patient can breathe coffer talk do not interfere with his or her attempts to expel the foreign body

As with adults encourage the child to continue coughing
administer supplement oxygen if needed and tolerated and provide transport to the emergency department
you should intervene only signs for severe airway obstruction develop such as a weak and effective cough cyanosis stridor absent air movement by decreasing level of consciousness

Removing a foreign body airway obstruction in a child 
responsive child 
if you determine a child older than one year has in airway obstruction then stand or kneel behind the child and provide abdominal thrusts in the same manner as an adult that use less force until the object is expelled of the child becomes unresponsive the child becomes unresponsive then follow the same steps as for the unresponsive adult to perform the abdominal thrust maneuver in a responsive child who is in the standing or sitting position follow these steps 
one kneel on one knee behind the child and circle both of your arms around the child's body prepared to give abdominal thrusts by placing your fist just above the patient's on umbilicus and well below the xiphoid processes process place your other hand 
over that fist to give the child abdominal thrusts in an upward direction 
avoid applying force to the lower rib cage or sternum 
three repeat this technique until the child expels the foreign body or becomes unresponsive for if the child becomes unresponsive position the child on a hard surface and immediately call for help or send someone to call for help 
five perform 30 chest compressions 15 compressions of two rescuers are present using the same landmark as you would for CPR do not check for a pulse before performing chest compressions 
six open the airway and look inside the mouth he could see an object that can be easily removed and attempt to remove it with your fingers 
and attempt to ventilate if you do not see an object and resume chest compressions 
seven repeat steps five and six until the obstruction is relieved or ALS providers take over 

if you managed to clear airway obstruction and unresponsive child but he or she still is no spontaneous breathing air circulation and perform CPR compressions in ventilations and apply the ADD as soon as it is available 


Unresponsive child if a child older than one year with airway obstruction becomes unresponsive he or she is managed in the same manner as an adult

The steps for moving a foreign body airway obstruction and unresponsive child:

1 Take Standard Precautions Carefully Pl. the child in the supine position on a firm flat surface  to perform 30 chest compressions 15 compressions of two rescuers are present using the same landmark as you would for CPR 
lower half of the sternum do not check for a pulse before performing chest compressions 
three open the airway and look inside the mouth 
for if you see an object that can be easily removed and remove it with your fingers and attempt to ventilate five if you do not see an object and resume chest compressions six repeat the sequence of chest compressions opening the airway and looking inside the mouth until the obstruction is relieved there ALS providers take over 


Removing a foreign body airway obstruction in infants 

Responsive infants 
do not use abdominal thrusts on a responsive infant with an airway obstruction because of the risk of injury to the immature organs of the abdomen instead perform back slaps and chest thrusts to try to clear severe airway obstruction 
in a responsive infant as follows 
one hold the infant face down with the body resting on your forearm support the infants jaw and face 
with your hand and keep the head lower than the rest of the body to deliver five back slaps between the shoulder blades using the heel of your hand 
through three place your free hand behind the infants head and back 
and turn the infant face up on your other forearm and thigh sandwiching the infants body between your two hands and arms the infant's head should remain below the level of the body or give five quick chest 
thrusts in the same manner and location is chest compressions using twos fingers placed on the lower half of the sternum for longer 
for larger infants or if you have small hands you can perform the step by placing the infant in your lap and turning the infant's whole body as a unit between back slaps and chest thrusts 

Five check the airway if you can see the foreign body then remove it if not then repeat the cycle as often as necessary six if the infant becomes unresponsive then begin CPR and follow the same sequence as for a child and adult 

Unresponsive infants 
if the infant becomes unresponsive there your attempts to relieve in airway obstruction to perform CPR study with chest compressions do not check for a pulse before starting compressions 
open the airway and looking 
and look in the mouth if you see an object they can easily remove the remove it with your finger and attempt to ventilate you do not see an object and resume chest compressions 

Continue sequence of chest compressions opening the airway and looking inside the mouth until the obstruction is relieved there ALS providers take over 

Whenever you assist the patient remember that his or in some patient apps that in some cases family members may experience a psychological crisis that turns to medical crisis and may become patients themselves 
\end{comment}
\end{document}
