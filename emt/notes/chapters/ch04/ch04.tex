\documentclass[../../EMT-169.tex]{subfiles}

\begin{document}
\setcounter{chapter}{3}
\label{ch:chapter4}
\clearpage
	
% Glossary acronym entries %
	\newacronym{fcc}{FCC}{Federal Communications Commission}
	\newacronym{mdt}{MDT}{mobile data terminal}
	\newacronym{pcr}{PCR}{patient care report}
	\newacronym{uhf}{UHF}{ultra-high frequency}
	\newacronym{vhf}{VHF}{very high frequency}
	\newacronym{voip}{VoIP}{Voice over Internet Protocol}
	
% Glossary entries
	\newglossaryentry{base_station}
	{
		name=base station,
		description={any radio hardware containing a transmitter and receiver that is located in a fixed place}
	}
	
	\newglossaryentry{cellular_telephone}
	{
		name=cellular telephone,
		description={a low-power portable radio that communicates through an interconnected series of repeater stations called “cells”}
	}
	
	\newglossaryentry{channel}
	{
		name=channel,
		description={an assigned frequency or frequencies that are used to carry voice and/or data communications}
	}
	
	\newglossaryentry{closed_ended_questions}
	{
		name=closed-ended questions,
		description={questions that can be answered in short bursts single word responses}
	}
	
	\newglossaryentry{communication}
	{
		name=communication,
		description={the transmission of information to another person— verbally or through body language}
	}
	
	\newglossaryentry{cultural_imposition}
	{
		name=cultural imposition,
		description={when one person poses his or her beliefs, values, and practices on another because he or she believe his or her ideals are superior}
	}
	
	\newglossaryentry{dedicated_line}
	{
		name=dedicated line,
		description={a special telephone line that is used for a specific point-to-point communications; also known as a hotline}
	}
	
	\newglossaryentry{documentation}
	{
		name=documentation,
		description={the recorded portion of the EMTs patient interaction either written or electronic.  This becomes part of the patient’s permanent medical record}
	}
	
	\newglossaryentry{duplex}
	{
		name=duplex,
		description={the ability to transmit and receive simultaneously}
	}
	
	\newglossaryentry{ethnocentrism}
	{
		name=ethnocentrism,
		description={when a person considers his or her own cultural values is more important when interacting with people of different culture}
	}
	
	\newglossaryentry{federal_communications_commission}
	{
		name=\acrfull{fcc},
		description={the federal agency that is jurisdiction over interstate and international telephone and telegraph services and cyclic medications, all of which may involve \acrshort{ems} activity}
	}
	
	\newglossaryentry{hotline}
	{
		name=hotline,
		description={see: dedicated line}
	}
	
	\newglossaryentry{interoperable_communication_system}
	{
		name=interoperable communication system,
		description={a communication system that uses \acrfull{voip} technology to allow multiple agencies to communicate and transmit data}
	}
	
	\newglossaryentry{med_channels}
	{
		name=MED channels,
		description={VHF and UHF channels that the Federal Communications Commission has designated exclusively for \acrshort{ems} use}
	}
	
	\newglossaryentry{mobile_data_terminal}
	{
		name=\acrfull{mdt},
		description={a small computer terminal inside the ambulance that directly receives data from the dispatch center}
	}
	
	\newglossaryentry{multiplex}
	{
		name=multiplex,
		description={the ability to transmit audio and data signals through the use of more than one communications channel}
	}
	
	\newglossaryentry{noise}
	{
		name=noise,
		description={anything that dampens obscures the true meaning of a message}
	}
	
	\newglossaryentry{open_ended_questions}
	{
		name=open-ended questions,
		description={questions for which the patient must provide detail to give an answer}
	}
	
	\newglossaryentry{paging}
	{
		name=paging,
		description={the use of a radio signal and a voice or digital message that is transmitted to pagers ("beepers") or desktop monitor radios}
	}
	
	\newglossaryentry{patient_care_report}
	{
		name=\acrfull{pcr},
		description={the legal document used to record all patient care activities.  This report has direct patient care functions but also administrative and quality control functions.  \acrshort{pcr}s are also known as prehospital care reports}
	}
	
	\newglossaryentry{prehospital_care_reports}
	{
		name=prehospital care reports,
		description={see: \acrfull{pcr}}
	}
	
	\newglossaryentry{protocols}
	{
		name=protocols,
		description={see: standing orders}
	}
	
	\newglossaryentry{rapport}
	{
		name=rapport,
		description={a trusting relationship that you build with your patient}
	}
	
	\newglossaryentry{repeater}
	{
		name=repeater,
		description={a special base station radio that receives messages and signals on one frequency and then automatically re-transmits them on a second frequency}
	}
	
	\newglossaryentry{scanner}
	{
		name=scanner,
		description={a radio receiver that searches or "scans" across several frequencies until the message completed; the process is then repeated}
	}
	
	\newglossaryentry{simplex}
	{
		name=simplex,
		description={single-frequency radio; transmissions can occur in either direction but not simultaneously in both; when one party transmits the other can only receive, and the party that is transmitting is unable to receive}
	}
	
	\newglossaryentry{standing_orders}
	{
		name=standing orders,
		description={written documents, signed by the EMS systems adequate director, that outline specific directions, permissions, and sometimes prohibitions regarding patient care; also called protocols}
	}
	
	\newglossaryentry{telemetry}
	{
		name=telemetry,
		description={a process in which electronic signals are converted into coded, audible signals; these signals can then be transmitted by radio or telephone to a receiver with a decoder at the hospital}
	}
	
	\newglossaryentry{therapeutic_communication}
	{
		name=therapeutic communication,
		description={verbal and nonverbal indication techniques that encourage patients to express their feelings and to achieve a positive relationship}
	}
	
	\newglossaryentry{trunking}
	{
		name=trunking,
		description={telecommunication systems that allow computer to maximize utilization of a group of frequencies}
	}
	
	\newglossaryentry{ultra_high_frequency}
	{
		name=\acrfull{uhf},
		description={radio frequencies between 300 and 3,000 MHz}
	}
	
	\newglossaryentry{very_high_frequency}
	{
		name=\acrfull{vhf},
		description={radio frequencies between 30 and 300 MHz; the VHF spectrum is further divided into “high” and “low” bands}
	}


\chapter{Communication and Documentation}

\subsection*{Abbreviations}
\begin{description}[leftmargin=!,labelwidth=\widthof{\bfseries ABCDE}]
	\item [\acrshort{fcc}] 		\acrlong{fcc}
	\item [\acrshort{mdt}] 		\acrlong{mdt}
	\item [\acrshort{pcr}] 		\acrlong{pcr}
	\item [\acrshort{uhf}] 		\acrlong{uhf}
	\item [\acrshort{vhf}] 		\acrlong{vhf}
	\item [\acrshort{voip}] 	\acrlong{voip}
\end{description}

\subsection*{Definitions}
\begin{description}	
	\item [\gls{base_station}]                      \glsdesc{base_station}
	\item [\gls{cellular_telephone}]                \glsdesc{cellular_telephone}
	\item [\gls{channel}]                           \glsdesc{channel}
	\item [\gls{closed_ended_questions}]        	\glsdesc{closed_ended_questions}
	\item [\gls{communication}]        				\glsdesc{communication}
	\item [\gls{cultural_imposition}]        		\glsdesc{cultural_imposition}
	\item [\gls{dedicated_line}]        			\glsdesc{dedicated_line}
	\item [\gls{documentation}]        				\glsdesc{documentation}
	\item [\gls{duplex}]        					\glsdesc{duplex}
	\item [\gls{ethnocentrism}]        				\glsdesc{ethnocentrism}
	\item [\gls{federal_communications_commission}]	\glsdesc{federal_communications_commission}
	\item [\gls{hotline}]        					\glsdesc{hotline}
	\item [\gls{med_channels}]        				\glsdesc{med_channels}
	\item [\gls{multiplex}]        					\glsdesc{multiplex}
	\item [\gls{noise}]        						\glsdesc{noise}
	\item [\gls{open_ended_questions}]        		\glsdesc{open_ended_questions}
	\item [\gls{paging}]        					\glsdesc{paging}
	\item [\gls{patient_care_report}]        		\glsdesc{patient_care_report}
	\item [\gls{prehospital_care_reports}]        	\glsdesc{prehospital_care_reports}
	\item [\gls{protocols}]        					\glsdesc{protocols}
	\item [\gls{rapport}]        					\glsdesc{rapport}
	\item [\gls{repeater}]        					\glsdesc{repeater}
	\item [\gls{scanner}]        					\glsdesc{scanner}
	\item [\gls{simplex}]        					\glsdesc{simplex}
	\item [\gls{standing_orders}]        			\glsdesc{standing_orders}
	\item [\gls{telemetry}]        					\glsdesc{telemetry}
	\item [\gls{therapeutic_communication}]        	\glsdesc{therapeutic_communication}
	\item [\gls{trunking}]        					\glsdesc{trunking}
	\item [\gls{ultra_high_frequency}]       		\glsdesc{ultra_high_frequency}
	\item [\gls{very_high_frequency}]        		\glsdesc{very_high_frequency}
\end{description}\hfill \\

%\afterpage{%
%\clearpage
%\subsection{Potential Test Questions}
%\begin{outline}[enumerate]
%	\1 Why do we need a template here?
%	\1[] Because LaTeX is stupid so we have to.
%\end{outline}
%}

\clearpage
\end{document}